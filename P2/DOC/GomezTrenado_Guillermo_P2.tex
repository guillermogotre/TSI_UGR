\documentclass{article}
% pre\'ambulo

\usepackage{lmodern}
\usepackage[T1]{fontenc}
\usepackage[spanish,activeacute]{babel}
\usepackage{mathtools}
\usepackage[utf8]{inputenc}
\PassOptionsToPackage{hyphens}{url}\usepackage{hyperref}

\expandafter\def\expandafter\UrlBreaks\expandafter{\UrlBreaks%  save the current one
  \do\a\do\b\do\c\do\d\do\e\do\f\do\g\do\h\do\i\do\j%
  \do\k\do\l\do\m\do\n\do\o\do\p\do\q\do\r\do\s\do\t%
  \do\u\do\v\do\w\do\x\do\y\do\z\do\A\do\B\do\C\do\D%
  \do\E\do\F\do\G\do\H\do\I\do\J\do\K\do\L\do\M\do\N%
  \do\O\do\P\do\Q\do\R\do\S\do\T\do\U\do\V\do\W\do\X%
  \do\Y\do\Z}

\usepackage{listings}
%\usepackage{listingsutf8}
%\usepackage[spanish]{babel}

\lstset{
	%inputencoding=utf8/latin1,
	literate=%
         {á}{{\'a}}1
         {í}{{\'i}}1
         {é}{{\'e}}1
         {ý}{{\'y}}1
         {ú}{{\'u}}1
         {ó}{{\'o}}1
         {ě}{{\v{e}}}1
         {š}{{\v{s}}}1
         {č}{{\v{c}}}1
         {ř}{{\v{r}}}1
         {ž}{{\v{z}}}1
         {ď}{{\v{d}}}1
         {ť}{{\v{t}}}1
         {ň}{{\v{n}}}1                
         {ů}{{\r{u}}}1
         {Á}{{\'A}}1
         {Í}{{\'I}}1
         {É}{{\'E}}1
         {Ý}{{\'Y}}1
         {Ú}{{\'U}}1
         {Ó}{{\'O}}1
         {Ě}{{\v{E}}}1
         {Š}{{\v{S}}}1
         {Č}{{\v{C}}}1
         {Ř}{{\v{R}}}1
         {Ž}{{\v{Z}}}1
         {Ď}{{\v{D}}}1
         {Ť}{{\v{T}}}1
         {Ň}{{\v{N}}}1                
         {Ů}{{\r{U}}}1,
	language=bash,
	%basicstyle=\ttfamily,
    columns=fullflexible,
    frame=single,
    breaklines=true,
    postbreak=\mbox{{$\hookrightarrow$}\space},
    mathescape=true,
    morekeywords={types,predicates,action,parameters,precondition,effect, objects, init, goal, functions, metric},
}

\usepackage{graphicx}
\graphicspath{ {screens/} }

\PassOptionsToPackage{hyphens}{url}\usepackage[hyphens]{url}

\usepackage{amssymb}
\usepackage{stmaryrd}

\usepackage{enumitem}

% macros 
\newcommand{\img}[1]{
\noindent\makebox[0.9\textwidth][c]{\includegraphics[width=0.9\textwidth,]{#1}}%
}

% title
\title{Técnicas de los sistemas inteligentes\\
Práctica 2} 

\author{Guillermo G\'omez Trenado | 77820354-S}

\begin{document}
% cuerpo del documento

\maketitle

\section*{Decisiones generales}
	
	Hay algunas decisiones que son transversales a todos los problemas: he sacrificado la versatilidad de las acciones por su legibilidad, es decir, en funciones donde la acción se realiza sobre un tipo con subtipos, como desplazarse o tirar un objeto al suelo, he definido sendas acciones para las distintas situaciones conflictivas, pues si no tenía que definir variables sólo de comprobación en los parámetros que eran posteriormente imprimidas por pantalla y no tenían ningún significado ni sentido, pero FF las necesita para construir los estados de verdad; en segundo lugar, y de forma parecida al anterior, he creado el predicado \emph{handFull} y \emph{bagFull} a fin de evitar tener que crear otro parámetro que será imprimido en pantalla y utilizarlo para el \emph{for all} sobre \emph{carriesHand} y \emph{carriesBag} respectivamente.
	
	Todos los resultados de los problemas están comprobados y son correctos, pero sólo comento aquellos que puedan resultar interesantes. El resto de salidas se encuentran al final del PDF con el nombre del problema para facilitar la lectura. 
	
\section{Definición general del modelo y el problema}

\subsection{Objetos del mundo}

	En este primer ejercicio la motivación es evidente, he definido un único tipo de personaje y un único tipo de regalo, y después, los tipos específicos serán instancias del tipo genérico, definiendo sus propiedades posteriormente en la definición del problema. \emph{Orientation} por otro lado, define los cuatro puntos cardinales y se definen en el problema.
	
\begin{lstlisting}
(:types
	character
	gift
	robot character gift - locatable
	orientation
	location
)
\end{lstlisting}

\subsection{Predicados}

	Lo más interesante en este apartado es el predicado \emph{link} que es direccionado y relaciona no sólo dos localizaciones sino la orientación de la segunda respecto a la primera, esto lo utilizaremos posteriormente junto a \emph{facing} para comprobar si un robot está en la dicrreción correcta para seguir un camino. \emph{Right} y \emph{Left} expresan la relación de dos puntos cardinales, es decir, norte está a la izquierda de este y este a la derecha del primero. \emph{hasGift} define si un personaje tiene un objeto o no, posteriormente lo cambiaremos por un valor numérico.

\begin{lstlisting}
(:predicates
	(link ?l1 - location ?l2 - location ?o - orientation)
    (at ?x - locatable ?l - location)
    (facing ?r - robot ?o -orientation)
    (right ?o1 -orientation ?o2 -orientation)
    (left ?o1 -orientation ?o2 -orientation)
    (carries ?r -robot ?o -gift)
    (handEmpty ?r -robot)
    (hasGift ?c -character)
)
\end{lstlisting}

\subsection{Acciones}

Esta acción es sencilla, comprueba la posición, la orientación y un enlace con tales características, tras esto falsea la anterior posición y hace verdadera la nueva.

\begin{lstlisting}
(:action GOTO
    :parameters (?r - robot ?o -orientation ?l1 -location ?l2 -location)
    :precondition (and 
        (at ?r ?l1)
        (facing ?r ?o)
        (link ?l1 ?l2 ?o)
    )
    :effect (and 
        (not (at ?r ?l1))
        (at ?r ?l2)
    )
)
\end{lstlisting}

Aquí las precondiciones sólo sirven para obtener los valores con los que definir la nueva orientación.

\begin{lstlisting}
(:action TURN-RIGHT
    :parameters (?r -robot ?o1 -orientation ?o2 -orientation)
    :precondition (and 
        (facing ?r ?o1)
        (right ?o1 ?o2)
    )
    :effect (and 
        (not (facing ?r ?o1))
        (facing ?r ?o2)
    )
)

(:action TURN-LEFT
    :parameters (?r -robot ?o1 -orientation ?o2 -orientation)
    :precondition (and 
        (facing ?r ?o1)
        (left ?o1 ?o2)
    )
    :effect (and 
        (not (facing ?r ?o1))
        (facing ?r ?o2)
    )
)
\end{lstlisting}

Comprobamos que tenga la mano vacía, y en caso de ser así cogemos el objeto, falseando la actual posición del objeto.

\begin{lstlisting}
(:action PICK
    :parameters (?r -robot ?l -location ?g -gift)
    :precondition (and 
        (handEmpty ?r)
        (at ?r ?l)
        (at ?g ?l)
    )
    :effect (and 
        (not (handEmpty ?r))
        (not (at ?g ?l))
        (carries ?r ?g)
    )
)
\end{lstlisting}

La única restricción es que lleve el objeto en la mano, después actualizamos el estado de la mano y la nueva posición del objeto. Nótese que no falseo \emph{carries} porque no nos interesa, las comprobaciones siempre se realizan con \emph{handEmpty}.

\begin{lstlisting}
(:action DROP
    :parameters (?r -robot ?l -location ?g -gift)
    :precondition (and 
        (not (handEmpty ?r))
        (carries ?r ?g)
        (at ?r ?l)
    )
    :effect (and 
        (handEmpty ?r)
        (at ?g ?l)
    )
)
\end{lstlisting}

Si se encuentran en la misma posición y el robot lleva el objeto en la mano se actualiza el estado de la mano y el del personaje.

\begin{lstlisting}
(:action GIVE
    :parameters (?r -robot ?l -location ?g -gift ?c -character)
    :precondition (and 
        (at ?r ?l)
        (at ?c ?l)
        (not (handEmpty ?r))
        (carries ?r ?g)
    )
    :effect (and 
        (handEmpty ?r)
        (hasGift ?c)
    )
)
\end{lstlisting}

\subsection{Problema planteado}

Definimos las 25 casillas, los personajes, los regalos, el robot y las orientaciones posibles.

\begin{lstlisting}
(:objects 
    loc1 loc2 loc3 loc4 loc5 loc6 loc7 loc8 loc9 loc10 loc11 loc12 loc13 loc14 loc15 loc16 loc17 loc18 loc19 loc20 loc21 loc22 loc23 loc24 loc25 - location
    princesa principe bruja profesor leonardo - character
    manzana rosa algoritmo oro oscar - gift
    robot1 - robot
    norte sur este oeste - orientation
)
\end{lstlisting}

Los caminos, y posiciones los definimos de la siguiente manera. Nótese que por cada camino aparecen dos entradas, una en un sentido y otra en el opuesto, cada uno con su correspondiente relación de orientación.
\begin{lstlisting}
(:init
    ;map
    (link loc1 loc2 este)
    (link loc2 loc1 oeste)
    ...
    (at princesa loc1)
    ...
    (at algoritmo loc19)
    ...    
\end{lstlisting}

Quedando de la siguiente manera

\img{img/map1}

Las relaciones espaciales 
\begin{lstlisting}
	(right norte este)
    ...
    (left norte oeste)
    ...
\end{lstlisting}

Y por último el estado incial del robot, el predicado \emph{handEmpty} pasará posteriormente a \emph{handFull} y así nos ahorraremos tener que definirlo en el problema.

\begin{lstlisting}
	(at robot1 loc1)
    (facing robot1 norte)
    (handEmpty robot1)
\end{lstlisting}

Finalmente el objetivo lo definimos sencillamente como:

\begin{lstlisting}
(:goal (and
        (hasGift princesa)
        (hasGift principe)
        (hasGift bruja)
        (hasGift profesor)
        (hasGift leonardo)
    )
)
\end{lstlisting}

Nótese que controlo que tenga al menos un regalo pero no cuántos, posteriormente modificaremos esto en el ejercicio 5 añadiendo una función.

\section{Coste de desplazamiento}

\subsection{Modificar el dominio}

	Este cambio es sencillo, sólo tenemos que añadir las funciones y modificar \emph{GOTO}
\begin{lstlisting}
(:functions 
    (pathLength ?r -robot)
    (linkLength ?l1 -location ?l2 -location)
)
...
(:action GOTO
    ...
    :effect (and 
        ...
        (increase (pathLength ?r) (linkLength ?l1 ?l2))
    )
)
\end{lstlisting}

\subsection{Extender el problema (1)}

	Lo único que he hecho en este paso es añadir por cada camino una distancia, todas con longitud 1.
\begin{lstlisting}
(:init
	...
	(= (linkLength loc1 loc2) 1)
	(= (linkLength loc2 loc1) 1)
	...
\end{lstlisting}

	Tras hacer esto podemos optimizar el plan para reducir la distancia del camino
\begin{lstlisting}
	(:metric minimize (pathLength robot1))
\end{lstlisting}

	Ejecutando el programa de la siguiente manera, nótese que no es el óptimo pero h=1 daba unas ejecuciones demasiado largas.
	
\begin{lstlisting}	
./Metric-FF/ff -d . -o ej2_domain.pddl -f ej2_problem.pddl -O -g 1 -h 2
\end{lstlisting}

	No dibujo el nuevo mapa con la longitud de los caminos pues es el mismo con 1s en los arcos.

\subsection{Extender el problema (2)}
	
	En la segunda versión del problema actualicé el peso a 100 entre loc1 y loc6 ---que era el primer camino que tomaba--- para ver que efectivamente evitaba ese camino e iba por loc1-loc2, y así lo hizo.

\section{Distintos tipos de zonas y mochila}

\subsection{Modificar el dominio para zonas}

Voy a añadir un tipo \emph{wear} que junto a \emph{gift} anteriormente definido son de tipo \emph{takeable}, así los métodos \emph{PICK} y \emph{DROP} pasan a tener como parámetro un \emph{takeable} en vez de un \emph{gift} y \emph{GIVE} sólo admite un \emph{gift}. Además añadimos un nuevo tipo para el tipo de suelo \emph{groundKind} y predicados para gestionar el tipo de suelo \emph{isKind}, si requiere de ropa especial \emph{isSpecial} y qué ropa necesita \emph{needs}. Nótese que isSpecial sólo se aplica a agua y bosque, mientras que tierra y piedra tienen \emph{isKind} pero no \emph{isSpecial}.

\begin{lstlisting}
(:types
	...
	gift wear - takeable
	robot character takeable - locatable
	groundKind
)

(:predicates
	...
	(isSpecial ?l -location)
    (isKind ?l -location ?gk -groundKind)
    (needs ?gk -groundKind ?w -wear)

\end{lstlisting} 

Modificamos GOTO y añadimos GOTO-SPECIAL, podríamos añadir los parámetros a GOTO y hacer una función general pero entonces FF los rellena con basura cuando hacemos el \textbf{OR} y la salida en pantalla es poco legible.

\begin{lstlisting}
(:action GOTO
    :parameters (?r - robot ?o -orientation ?l1 -location ?l2 -location)
    :precondition (and
        (at ?r ?l1)
        (facing ?r ?o)
        (link ?l1 ?l2 ?o)
        (not (isSpecial ?l2))
    )
    :effect (and 
        (not (at ?r ?l1))
        (at ?r ?l2)
        (increase (pathLength ?r) (linkLength ?l1 ?l2))
    )
)
(:action GOTO-SPECIAL
    :parameters (?r - robot ?o -orientation ?l1 -location ?l2 -location ?gk -groundKind ?w -wear)
    :precondition (and 
        (at ?r ?l1)
        (facing ?r ?o)
        (link ?l1 ?l2 ?o)
        (isSpecial ?l2)
        (isKind ?l2 ?gk)
        (needs ?gk ?w)
        (carries ?r ?w)
    )
    :effect (and 
        (not (at ?r ?l1))
        (at ?r ?l2)
        (increase (pathLength ?r) (linkLength ?l1 ?l2))
    )
)
\end{lstlisting}

Además ahora en \emph{DROP} controlamos que no sea un terreno especial, de igual forma que en el caso anterior lo dividimos en dos acciones distintas por el mismo motivo. Hay dos pequeños errores en este ejercicio que no se solucionarán hasta el 6, el primero es que si sólo hay un objeto de cada tipo el programa funciona perfectamente, pero si hay varios \emph{wear} de un mismo tipo en el mapa sucede que no puede soltar el elemento \emph{wear} que está usando aunque tenga otro igual en la mochila, es un problema menor pero limita el espacio solución y no me di cuenta hasta 3 ejercicios más adelante al releer esta parte del código. El segundo es que por los parámetros de \emph{DROP-SPECIAL} no se pueden soltar en el bosque objeto de tipo gift, en el ejercicio se añadirá una tercera función para dejar los mensajes de salida significativos y la lógica correcta.

\begin{lstlisting}
(:action DROP
    :parameters (?r -robot ?l -location ?g -takeable)
    :precondition (and 
        (handFull ?r)
        (carries ?r ?g)
        (not (carriesBag ?r ?g))
        (at ?r ?l)
        (not (isSpecial ?l))
    )
    :effect (and 
        (not (handFull ?r))
        (not (carries ?r ?g))
        (at ?g ?l)
    )
)

(:action DROP-SPECIAL
    :parameters (?r -robot ?l -location ?g -wear ?gk -groundKind)
    :precondition (and 
        (handFull ?r)
        (carries ?r ?g)
        (not (carriesBag ?r ?g))
        (at ?r ?l)
        (isKind ?l ?gk)
        (not (needs ?gk ?g))
    )
    :effect (and 
        (not (handFull ?r))
        (not (carries ?r ?g))
        (at ?g ?l)
    )
)
\end{lstlisting}

\subsection{Modificar el dominio para mochila}

Aunque hemos adelantado algunas modificaciones del problema para la mochila vamos a verlas aquí en profundidad.

Definimos nuevos predicados. Como sucedía con el error del apartado anterior, aquí dejar implícitamente definida $\mathit{carriesHand}=\mathit{carries}\land\lnot\mathit{carriesBag}$ puede generar problemas cuando hay varios objetos \emph{wear} del mismo tipo, también lo soluciono en el ejercicio 6, de momento no general ninguna situación limitante porque los problemas que defino sólo tienen un objeto de cada tipo en el mapa.

\begin{lstlisting}
(:predicates
    ...
    (carries ?r -robot ?o -takeable)
    ;carriesHand -> carries and !carriesBag
    (carriesBag ?r -robot ?o -takeable)
    (handFull ?r -robot)
    (bagFull ?r -robot)
    ...
\end{lstlisting}

También actualizo todas las acciones donde comprobaba que llevara el objeto cambiando \emph{(carries ?r ?g)} por \emph{(AND (carries ?r ?g) (not (carriesBag ?r ?g))} para comprobar que efectivamente lo lleva en la mano. Esta situación genera que posteriormente, en el problema del ejercicio 4, no pueda dar un regalo si tiene otro del mismo tipo en la mochila, pero como ya he dicho se solucionará más adelante, conservo los errores por ahorrar tiempo por algo que no tendría interés didáctico ---ya lo soluciono más adelante--- y por reflejar la evolución del razonamiento incluso el equivocado.

Añadimos también los métodos para gestionar la mochila. Son dos métodos complementarios, y donde uno comprueba tener algo en la mano y la mochila vacía el otro lo contrario. Nótese que carries no hace referencia a llevar en la mano si no a llevar en general, comprobando con \emph{handFull} y \emph{bagFull} que nunca lleve más de dos objetos, tanto aquí como en \emph{PICK} y \emph{DROP}.

\begin{lstlisting}
(:action PUSH-BAG
    :parameters (?r -robot ?o -takeable)
    :precondition (and 
        (handFull ?r)
        (not (bagFull ?r))
        (carries ?r ?o)
    )
    :effect (and 
        (not (handFull ?r))
        (bagFull ?r)
        (carriesBag ?r ?o)
    )
)

(:action PULL-BAG
    :parameters (?r -robot ?o -takeable)
    :precondition (and 
        (not (handFull ?r))
        (bagFull ?r)
        (carries ?r ?o)
    )
    :effect (and 
        (handFull ?r)
        (not (bagFull ?r))
        (not (carriesBag ?r ?o))
    )
)
\end{lstlisting}

\subsection{Extender el problema}

Añadimos los nuevos objetos y sus predicados, nótese que mientras que tanto agua como bosque como precipicio son especial el último no tiene objeto que permita recorrerlo. Por el contrario, arena y piedra tienen tipo pero no son especiales y no requieren por tanto de objeto para pasar por él.

\begin{lstlisting}
(:objects 
	...
	bosque agua precipicio arena piedra - groundKind
    zapatilla bikini - wear
)
(:init
	...
	;#land conditions
    (needs agua bikini)
    (needs bosque zapatilla)
    
	;#special locs
    (isSpecial loc22)
    (isKind loc22 agua)
    (isSpecial loc23)
    (isKind loc23 bosque)

    (isSpecial loc20)
    (isKind loc20 precipicio)

    ;#not so special locs
    (isKind loc7 arena)
    (isKind loc2 piedra)
    
    ;#wear position
    (at bikini loc16)
    (at zapatilla loc22)
    
    (at algoritmo loc23)
\end{lstlisting}

Quedando así el mapa

\img{img/map3}

El único objetivo es que el profesor consiga un regalo, y es bonito ver la solución porque no puede tirar las zapatillas para coger el algoritmo, entonces tiene que coger el bikini, las zapatillas, volver a la casilla 21, cambiar el orden de los objetos soltar el bikini en la casilla 23, coger el algoritmo y dárselo al profesor. Veamos un fragmeno de la salida.

\begin{lstlisting}
7: GOTO ROBOT1 OESTE LOC17 LOC16
8: TURN-LEFT ROBOT1 OESTE SUR
9: TURN-LEFT ROBOT1 SUR ESTE
10: PICK ROBOT1 LOC16 BIKINI
11: GOTO ROBOT1 ESTE LOC16 LOC17
12: PUSH-BAG ROBOT1 BIKINI
13: TURN-RIGHT ROBOT1 ESTE SUR
14: GOTO-SPECIAL ROBOT1 SUR LOC17 LOC22 AGUA BIKINI
15: PICK ROBOT1 LOC22 ZAPATILLA
16: TURN-RIGHT ROBOT1 SUR OESTE
17: GOTO ROBOT1 OESTE LOC22 LOC21
18: TURN-RIGHT ROBOT1 OESTE NORTE
19: DROP ROBOT1 LOC21 ZAPATILLA
20: PULL-BAG ROBOT1 BIKINI
21: DROP ROBOT1 LOC21 BIKINI
22: TURN-RIGHT ROBOT1 NORTE ESTE
23: PICK ROBOT1 LOC21 ZAPATILLA
24: PUSH-BAG ROBOT1 ZAPATILLA
25: PICK ROBOT1 LOC21 BIKINI
26: GOTO-SPECIAL ROBOT1 ESTE LOC21 LOC22 AGUA BIKINI
27: GOTO-SPECIAL ROBOT1 ESTE LOC22 LOC23 BOSQUE ZAPATILLA
28: DROP-SPECIAL ROBOT1 LOC23 BIKINI BOSQUE
29: PICK ROBOT1 LOC23 ALGORITMO
30: GOTO ROBOT1 ESTE LOC23 LOC24
31: GOTO ROBOT1 ESTE LOC24 LOC25
32: GIVE ROBOT1 LOC25 ALGORITMO PROFESOR
\end{lstlisting}

\section{Puntos por regalos}
\subsection{Modificar el dominio}

Este cambio es más sencillo, sólo debemos añadir algunas funciones y modificar \emph{GIVE}.

\begin{lstlisting}
(:functions
    ...
    (pointsEarned ?r -robot)
    (pointsFor ?c -character ?g -gift)
)

(:action GIVE
    :parameters (?r -robot ?l -location ?g -gift ?c -character)
    :precondition (and 
        (at ?r ?l)
        (at ?c ?l)
        (handFull ?r)
        (carries ?r ?g)
        (not (carriesBag ?r ?g))
    )
    :effect (and 
        (not (handFull ?r))
        (not (carries ?r ?g))
        (hasGift ?c)
        (increase (pointsEarned ?r) (pointsFor ?c ?g))
    )
)
\end{lstlisting}
\subsection{Modificar el problema}

Añadimos la matriz de puntos al problema

\begin{lstlisting}
(:init
    ...
    ;points For
    (= (pointsFor leonardo oscar) 10)
    (= (pointsFor leonardo rosa) 1)
    (= (pointsFor leonardo manzana) 3)
    (= (pointsFor leonardo algoritmo) 4)
    (= (pointsFor leonardo oro) 5)
    ...
\end{lstlisting}

Modificamos los regalos y los objetivos. Ahora buscamos que todos los personajes tengan al menos un regalo, y la combinación de puntos sea de al menos 40, el espacio solución se reduce y como lo tengo configurado para optimizar el número de instrucciones con $g=1,h=5$ ya tarda 17.4 segundos en encontrar una solución. Al final del documento dejo constancia de las opciones de ejecución de cada problema.

\begin{lstlisting}
(:init
    ...
    ;gifts
    (at oscar loc7)
    (at oscar loc9)
    (at oscar loc14)
    (at manzana loc17)
    (at manzana loc23)
    (at manzana loc13)
    ...
)
(:goal 
	(and
        (hasGift princesa)
        (hasGift principe)
        (hasGift bruja)
        (hasGift profesor)
        (hasGift leonardo)
        (> (pointsEarned robot1) 40)
    )
)
\end{lstlisting}

\img{img/map4}

Finalmente llega a la conclusión de que el reparto que satisface la restricción es la manzana a la bruja, dos oscars a leonardo, el oscar a la princesa y sendas manzanas al profesor y al príncipe, sumando en total \textbf{44}.

\section{Bolsillo mágico de los personajes}

\subsection{Modificar el dominio}

Sólo tenemos que añadir una función que se inicialice con el número máximo de regalos para cada personaje y modificar \emph{GIVE} para comprobar que queda espacio en el bolsillo y actualizarlo. En este problema conservo \emph{hasGift}, para el ejercicio 7 ya lo elimino y trabajo únicamente con el número restante de regalos que caben.

\begin{lstlisting}
(:functions ;todo: define numeric functions here
	...
    (maxGifts ?c -character)
)

(:action GIVE
    :parameters (?r -robot ?l -location ?g -gift ?c -character)
    :precondition (and 
        (at ?r ?l)
        (at ?c ?l)
        (handFull ?r)
        (carries ?r ?g)
        (not (carriesBag ?r ?g))
        (> (maxGifts ?c) 0)
    )
    :effect (and 
        (not (handFull ?r))
        (not (carries ?r ?g))
        (hasGift ?c)
        (increase (pointsEarned ?r) (pointsFor ?c ?g))
        (decrease (maxGifts ?c) 1)
    )
)
\end{lstlisting}

\subsection{Modificar el problema}

	En este caso retomo el problema anterior eliminando el requisito de tener un regalo el príncipe  y prohibiendo la solución que nos daba el robot ---Leonardo sólo puede coger un regalo---. Definimos el tamaño máximo del bolsillo y el nuevo objetivo. Este problema es suficientemente complejo como para evaluar correctamente el functionamiento del programa, para los dos siguientes ejercicios sí habrá dos problemas distintos para poder evaluarlo de forma exhaustiva.

\begin{lstlisting}
(:init
	...
	;characters max
    (= (maxGifts princesa) 2)
    (= (maxGifts principe) 2)
    (= (maxGifts bruja) 2)
    (= (maxGifts profesor) 2)
    (= (maxGifts leonardo) 1)
	...
)
(:goal (and
        (hasGift princesa)
        (hasGift bruja)
        (hasGift profesor)
        (hasGift leonardo)
        (>= (pointsEarned robot1) 45)
    )
)   
\end{lstlisting}

Llega a la solución donde le da dos manzanas a la bruja, dos óscars a la princesa, otro óscar a Leonardo y la manzana restante al profesor, así consigue llegar a la única solución de exactamente \textbf{45} puntos.

\section{Dos jugadores cooperantes}

\subsection{Modificar el dominio}

Aquí sólo tenemos que añadir una nueva función para controlar el número total de puntos ---\emph{pointEarned} se queda como estaba---, así como modificar las acciones para actualizar este valor al entregar un regalo.

\begin{lstlisting}
(:functions ;todo: define numeric functions here
    ...
    (pointsEarned ?r -robot)
    (totalPoints)
    ...
)

(:action GIVE
    :parameters (?r -robot ?l -location ?g -gift ?c -character)
    :precondition (and 
        (at ?r ?l)
        (at ?c ?l)
        (handFull ?r)
        (carries ?r ?g)
        (not (carriesBag ?r ?g))
        (> (maxGifts ?c) 0)
    )
    :effect (and 
        (not (handFull ?r))
        (not (carries ?r ?g))
        ;(hasGift ?c)
        (increase (pointsEarned ?r) (pointsFor ?c ?g))
        (increase (totalPoints) (pointsFor ?c ?g))
        (decrease (maxGifts ?c) 1)
    )
)
\end{lstlisting}

\subsection{Modificar el problema (1)}

Para el problema, inicializamos el contador total a 0, añadimos un nuevo robot y definimos los puntos mínimos para cada uno.

\begin{lstlisting}
(:objects 
    ...
    robot1 robot2 - robot
    ...
)
(:init
	...
	;#points
    (= (totalPoints) 0)
    ;#robot1
    (at robot1 loc1)
    (facing robot1 norte)
    (= (pathLength robot1) 0)
    (= (pointsEarned robot1) 0)
    ;#robot2
    (at robot2 loc25)
    (facing robot2 norte)
    (= (pathLength robot2) 0)
    (= (pointsEarned robot2) 0)
)
(:goal (and
        (>= (pointsEarned robot1) 10)
        (>= (pointsEarned robot2) 10)
    )
)
\end{lstlisting}

Nos basamos en el problema anterior pero añadiendo un plus de dificultad, y es que para poder satisfacer el robot dos su cometido necesita la colaboración del primero, pues tiene que llevarle las zapatillas para poder coger el regalo del bosque, o las zapatillas y el bikini para salir de su islote y satisfacer su objetivo.

\img{img/map6}

Efectivamente así lo hace, veamos un fragmento de la salida.

\begin{lstlisting}
25: GOTO-SPECIAL ROBOT1 SUR LOC17 LOC22 AGUA BIKINI
26: TURN-LEFT ROBOT1 SUR ESTE
27: PICK ROBOT1 LOC22 ZAPATILLA
28: GOTO-SPECIAL ROBOT1 ESTE LOC22 LOC23 BOSQUE ZAPATILLA
29: GOTO ROBOT1 ESTE LOC23 LOC24
30: DROP ROBOT1 LOC24 ZAPATILLA
31: GOTO ROBOT2 OESTE LOC25 LOC24
32: PICK ROBOT2 LOC24 ZAPATILLA
33: PUSH-BAG ROBOT2 ZAPATILLA
34: GOTO-SPECIAL ROBOT2 OESTE LOC24 LOC23 BOSQUE ZAPATILLA
35: TURN-LEFT ROBOT2 OESTE SUR
36: TURN-LEFT ROBOT2 SUR ESTE
37: PICK ROBOT2 LOC23 MANZANA
38: GOTO ROBOT2 ESTE LOC23 LOC24
39: GOTO ROBOT2 ESTE LOC24 LOC25
40: GIVE ROBOT2 LOC25 MANZANA PROFESOR
41: PULL-BAG ROBOT1 BIKINI
42: TURN-RIGHT ROBOT2 ESTE SUR
43: TURN-RIGHT ROBOT2 SUR OESTE
44: GOTO ROBOT2 OESTE LOC25 LOC24
45: DROP ROBOT1 LOC24 BIKINI
46: PICK ROBOT2 LOC24 BIKINI
47: GOTO-SPECIAL ROBOT2 OESTE LOC24 LOC23 BOSQUE ZAPATILLA
48: GOTO-SPECIAL ROBOT2 OESTE LOC23 LOC22 AGUA BIKINI
49: TURN-RIGHT ROBOT2 OESTE NORTE
50: GOTO ROBOT2 NORTE LOC22 LOC17
51: GOTO ROBOT2 NORTE LOC17 LOC12
52: TURN-RIGHT ROBOT2 NORTE ESTE
53: DROP ROBOT2 LOC12 BIKINI
54: GOTO ROBOT2 ESTE LOC12 LOC13
55: GOTO ROBOT2 ESTE LOC13 LOC14
56: TURN-RIGHT ROBOT2 ESTE SUR
57: TURN-RIGHT ROBOT2 SUR OESTE
58: PICK ROBOT2 LOC14 OSCAR
59: GOTO ROBOT2 OESTE LOC14 LOC13
60: GIVE ROBOT2 LOC13 OSCAR LEONARDO
\end{lstlisting}

Tras conseguir ROBOT1 sus 10 puntos le lleva las zapatillas y el bikini a ROBOT2, primero le da las zapatillas, coge la manzana, se la da al profesor y después le da el ROBOT1 el bikini para poder salir de la isla y conseguir otros 10 puntos dándole el óscar a leonardo. Es prácticamente mágico.

\subsection{Modificar el problema (2)}

En este caso vamos a repetir un problema muy parecido al anterior pero cambiando los términos del objetivo.

\begin{lstlisting}
(:goal (and
        (>= (pointsEarned robot1) 5)
        (<= (pointsEarned robot1) 10)
        (>= (totalPoints) 15)
    )
)
\end{lstlisting}

Aquí podemos ver un situación muy interesante, aunque la solución del problema anterior ---que tardó 0.3 segundos en obtenerla--- satisface los requisitos de este objetivo, para este problema tarda 45 minutos porque no es tan fácil ver que tiene que llevarle las zapatillas y el bikini el ROBOT1 al ROBOT2 para poder satisfacer los objetivos. Por el contrario, si sólo pongo como objetivo \emph{(>= (totalPoints) 15)} lo resuelve de forma inmediata.

\section{Robots diferenciados}

\subsection{Corrigiendo los errores anteriores}

	Aquí ya están solucionados todos los problemas que hemos ido comentando en ejercicios anteriores. He eliminado el carriesHand deducido por uno explícitamente definido y en las acciones que lo requieran se comprueba si está en alguno de los dos sitios.

\begin{lstlisting}
(:predicates ;todo: define predicates here
    ...
    (carriesHand ?r -robot ?o -takeable)
    (carriesBag ?r -robot ?o -takeable)
    (handFull ?r -robot)
    (bagFull ?r -robot)
	...
)

(:action GOTO-SPECIAL
    :parameters (?r - robot ?o -orientation ?l1 -location ?l2 -location ?gk -groundKind ?w -wear)
    :precondition (and 
        (at ?r ?l1)
        (facing ?r ?o)
        (link ?l1 ?l2 ?o)
        (isSpecial ?l2)
        (isKind ?l2 ?gk)
        (needs ?gk ?w)
        (or
            (carriesHand ?r ?w)
            (carriesBag ?r ?w)
        )
    )
    :effect (and 
        (not (at ?r ?l1))
        (at ?r ?l2)
        (increase (pathLength ?r) (linkLength ?l1 ?l2))
    )
)
\end{lstlisting} 

Como ya avanzábamos hemos modificado la lógica del drop para permitir soltar correctamente \emph{takeables} en cualquier zona dependiendo de la legalidad de la acción y conservar los mensajes de salida significativos.

\begin{lstlisting}
(:action DROP
    :parameters (?r -robot ?l -location ?g -takeable)
    :precondition (and 
        (handFull ?r)
        (carriesHand ?r ?g)
        (at ?r ?l)
        (not (isSpecial ?l))
    )
    :effect (and 
        (not (handFull ?r))
        (not (carriesHand ?r ?g))
        (at ?g ?l)
    )
)

(:action DROP-SPECIAL-GIFT
    :parameters (?r -robot ?l -location ?g -gift ?gk -groundKind)
    :precondition (and 
        (handFull ?r)
        (carriesHand ?r ?g)
        (at ?r ?l)
        (isSpecial ?l)
    )
    :effect (and 
        (not (handFull ?r))
        (not (carriesHand ?r ?g))
        (at ?g ?l)
    )
)

(:action DROP-SPECIAL-WEAR
    :parameters (?r -robot ?l -location ?g -wear ?gk -groundKind)
    :precondition (and 
        (handFull ?r)
        (carriesHand ?r ?g)
        (at ?r ?l)
        (isKind ?l ?gk)
        (or
            (carriesBag ?r ?g)
            (not (needs ?gk ?g))
        )
    )
    :effect (and 
        (not (handFull ?r))
        (not (carriesHand ?r ?g))
        (at ?g ?l)
    )
)
\end{lstlisting}

\subsection{Modificar el modelo}

Ahora, para añadir al modelo la lógica de los robots con capacidades diferenciadas añadimos dos nuevos subtipos de robot, \emph{carrier} y \emph{giver}, el primero coge objetos del suelo y se los da a \emph{giver}, mientras que éste último sólo puede interactuar con humanos.

\begin{lstlisting}
(:types
	carrier giver - robot
    robot character takeable - locatable
)
\end{lstlisting}

Ahora tenemos que modificar \emph{PICK y GIVE}, y añadir un nuevo método, \emph{GIVE-COOP}. No modificamos ninguno de los tres \emph{DROP}, ambos robots pueden tirar algo al suelo, pero solo \emph{carrier} puede recogerlo.

\begin{lstlisting}
(:action PICK
    :parameters (?r - carrier ?l -location ?g -takeable)
    :precondition (and 
        (not (handFull ?r))
        (at ?r ?l)
        (at ?g ?l)
    )
    :effect (and 
        (handFull ?r)
        (not (at ?g ?l))
        (carriesHand ?r ?g)
    )
)
(:action GIVE
    :parameters (?r -giver ?l -location ?g -gift ?c -character)
    :precondition (and 
        (at ?r ?l)
        (at ?c ?l)
        (handFull ?r)
        (carriesHand ?r ?g)
        (> (maxGifts ?c) 0)
    )
    :effect (and 
        (not (handFull ?r))
        (not (carriesHand ?r ?g))
        (increase (pointsEarned ?r) (pointsFor ?c ?g))
        (decrease (maxGifts ?c) 1)
    )
)

(:action GIVE-COOP
    :parameters (?c -carrier ?g -giver ?l -location ?t -takeable)
    :precondition (and 
        (at ?c ?l)
        (at ?g ?l)
        (handFull ?c)
        (carriesHand ?c ?t)
        (not (handFull ?g))
    )
    :effect (and 
        (not (handFull ?c))
        (not (carriesHand ?c ?t))
        (handFull ?g)
        (carriesHand ?g ?t)
    )
)
\end{lstlisting}


\subsection{Modificar el problema (1)}

	Para el correcto funcionamiento sólo tenemos que definir ROBOT1 y ROBOT2 como tipos diferenciados.
	
\begin{lstlisting}
(:objects 
    robot1 - carrier
    robot2 - giver
)
\end{lstlisting}

	Para este problema, ya que ROBOT1 no puede recoger ni las zapatillas ni el bikini del suelo para poder salir de la isla, eliminamos el bosque y planteamos como objetivo que ROBOT2 le de un regalo a la bruja, es prácticamente como el problema anterior pero ahora el intercambio del bikini y las zapatillas debe ser con \emph{GIVE-COOP}.
	
\begin{lstlisting}
(:init
	;#special locs
    (isSpecial loc22)
    (isKind loc22 agua)

    (isSpecial loc20)
    (isKind loc20 precipicio)

    ;#not so special locs
    (isKind loc7 arena)
    (isKind loc2 piedra)
    
    ...
    
    ;#gifts
    (at oscar loc7)
    (at oscar loc9)
    (at oscar loc14)
    (at manzana loc17)
    (at manzana loc13)
	
	...
)
(:goal 
    (and
        (= (maxGifts bruja) 1)
    )
)   
\end{lstlisting}

\img{img/map7}

Veamos un fragmento de la salida.

\begin{lstlisting}
12: GOTO ROBOT1 OESTE LOC17 LOC16
13: TURN-LEFT ROBOT1 OESTE SUR
14: TURN-LEFT ROBOT1 SUR ESTE
15: PICK ROBOT1 LOC16 BIKINI
16: GOTO ROBOT1 ESTE LOC16 LOC17
17: PUSH-BAG ROBOT1 BIKINI
18: PICK ROBOT1 LOC17 MANZANA
19: TURN-RIGHT ROBOT1 ESTE SUR
20: GOTO-SPECIAL ROBOT1 SUR LOC17 LOC22 AGUA BIKINI
21: TURN-LEFT ROBOT1 SUR ESTE
22: GOTO ROBOT1 ESTE LOC22 LOC23
23: GIVE-COOP ROBOT1 ROBOT2 LOC23 BIKINI
24: PULL-BAG ROBOT1 BIKINI
25: PUSH-BAG ROBOT2 BIKINI
26: GIVE-COOP ROBOT1 ROBOT2 LOC23 MANZANA
27: GOTO-SPECIAL ROBOT2 OESTE LOC23 LOC22 AGUA BIKINI
28: GOTO ROBOT2 OESTE LOC22 LOC21
29: GIVE ROBOT2 LOC21 MANZANA BRUJA
\end{lstlisting}

	Efectivamente ROBOT1 coge la manza y el bikini y se lo lleva a ROBOT2, y éste último le entrega la manzana a la bruja.
\subsection{Modificar el problema (2)}

	En esta versión, sobre el mapa anterior, en las mismas condiciones cambiamos el objetivo para que todos los personajes reciban exactamente 1 regalo. Tengo que añadir otro bikini dentro del islote para que ROBOT1 pueda salir con ROBOT2 y ayudarle a entregar los regalos.
	
\begin{lstlisting}
(:init
	....	
	;wears
    (at bikini loc16)
    (at bikini loc24)
    (at zapatilla loc22)
    ...
)
\end{lstlisting}

\img{img/map7_2}

Efectivamente traza el plan de 124 pasos que adjunto al final.

\subsection{Modificar el problema (3)}
	Finalmente, y para regodearnos en nuestro éxito vamos a definir un problema combinando varias cosas de las ya implementadas.
	
\begin{lstlisting}
(:init
	...
	;gifts
    (at oscar loc7)
    (at manzana loc9)
    (at rosa loc17)
    (at algoritmo loc23)
    (at oro loc13)
    ...
)

(:goal (and
        (= (maxGifts princesa) 1)
        (= (maxGifts principe) 1)
        (= (maxGifts bruja) 1)
        (= (maxGifts profesor) 1)
        (= (maxGifts leonardo) 1)
        (= (totalPoints) 50)
    )
)
\end{lstlisting}
	
	No ha sido precisamente rápido pero lo ha obtenido.
	
\section{Salidas}
\subsection*{1.1}
\begin{lstlisting}
ff: parsing domain file
domain 'BELKAN' defined
 ... done.
ff: parsing problem file
problem 'EJ1' defined
 ... done.


no metric specified. plan length assumed.

checking for cyclic := effects --- OK.

ff: search configuration is  best-first on 1*g(s) + 5*h(s) where
    metric is  plan length

advancing to distance:   23
                         22
                         20
                         19
                         18
                         17
                         16
                         14
                         13
                         12
                         11
                         10
                          9
                          8
                          7
                          6
                          5
                          4
                          3
                          2
                          1
                          0

ff: found legal plan as follows

step    0: PICK ROBOT1 LOC1 PRINCESA
        1: TURN-RIGHT ROBOT1 NORTE ESTE
        2: GOTO ROBOT1 ESTE LOC1 LOC2
        3: TURN-RIGHT ROBOT1 ESTE SUR
        4: GOTO ROBOT1 SUR LOC2 LOC7
        5: GOTO ROBOT1 SUR LOC7 LOC12
        6: TURN-LEFT ROBOT1 SUR ESTE
        7: GOTO ROBOT1 ESTE LOC12 LOC13
        8: TURN-RIGHT ROBOT1 ESTE SUR
        9: DROP ROBOT1 LOC13 PRINCESA
       10: PICK ROBOT1 LOC13 ORO
       11: GIVE ROBOT1 LOC13 PRINCESA PRINCESA
       12: PICK ROBOT1 LOC13 PRINCESA
       13: GIVE ROBOT1 LOC13 ORO LEONARDO
       14: PICK ROBOT1 LOC13 LEONARDO
       15: TURN-LEFT ROBOT1 SUR ESTE
       16: GOTO ROBOT1 ESTE LOC13 LOC14
       17: GOTO ROBOT1 ESTE LOC14 LOC15
       18: TURN-RIGHT ROBOT1 ESTE SUR
       19: GOTO ROBOT1 SUR LOC15 LOC20
       20: GOTO ROBOT1 SUR LOC20 LOC25
       21: TURN-RIGHT ROBOT1 SUR OESTE
       22: GIVE ROBOT1 LOC25 ORO PROFESOR
       23: PICK ROBOT1 LOC25 PROFESOR
       24: TURN-RIGHT ROBOT1 OESTE NORTE
       25: GOTO ROBOT1 NORTE LOC25 LOC20
       26: GOTO ROBOT1 NORTE LOC20 LOC15
       27: GOTO ROBOT1 NORTE LOC15 LOC10
       28: GOTO ROBOT1 NORTE LOC10 LOC5
       29: TURN-LEFT ROBOT1 NORTE OESTE
       30: GIVE ROBOT1 LOC5 PROFESOR PRINCIPE
       31: PICK ROBOT1 LOC5 PRINCIPE
       32: GOTO ROBOT1 OESTE LOC5 LOC4
       33: GOTO ROBOT1 OESTE LOC4 LOC3
       34: TURN-LEFT ROBOT1 OESTE SUR
       35: GOTO ROBOT1 SUR LOC3 LOC8
       36: TURN-RIGHT ROBOT1 SUR OESTE
       37: GOTO ROBOT1 OESTE LOC8 LOC7
       38: TURN-LEFT ROBOT1 OESTE SUR
       39: GOTO ROBOT1 SUR LOC7 LOC12
       40: GOTO ROBOT1 SUR LOC12 LOC17
       41: GOTO ROBOT1 SUR LOC17 LOC22
       42: TURN-RIGHT ROBOT1 SUR OESTE
       43: GOTO ROBOT1 OESTE LOC22 LOC21
       44: GIVE ROBOT1 LOC21 PROFESOR BRUJA


time spent:    0.00 seconds instantiating 1820 easy, 0 hard action templates
               0.00 seconds reachability analysis, yielding 296 facts and 1820 actions
               0.00 seconds creating final representation with 296 relevant facts, 0 relevant fluents
               0.00 seconds computing LNF
               0.00 seconds building connectivity graph
               0.02 seconds searching, evaluating 367 states, to a max depth of 0
               0.02 seconds total time
\end{lstlisting}
\subsection*{2.1}
\begin{lstlisting}

ff: parsing domain file
domain 'BELKAN' defined
 ... done.
ff: parsing problem file
problem 'EJ1' defined
 ... done.


metric established (normalized to minimize): ((1.00*[RF0](PATHLENGTH_ROBOT1)) - () + 0.00)

checking for cyclic := effects --- OK.

ff: search configuration is  best-first on 1*g(s) + 2*h(s) where
    metric is ((1.00*[RF0](PATHLENGTH_ROBOT1)) - () + 0.00)

advancing to distance:   23
                         22
                         21
                         19
                         18
                         17
                         16
                         15
                         14
                         13
                         12
                         11
                         10
                          9
                          8
                          7
                          6
                          5
                          4
                          3
                          2
                          1
                          0

ff: found legal plan as follows

step    0: TURN-RIGHT ROBOT1 NORTE ESTE
        1: TURN-RIGHT ROBOT1 ESTE SUR
        2: GOTO ROBOT1 SUR LOC1 LOC6
        3: TURN-LEFT ROBOT1 SUR ESTE
        4: GOTO ROBOT1 ESTE LOC6 LOC7
        5: PICK ROBOT1 LOC7 OSCAR1
        6: TURN-LEFT ROBOT1 ESTE NORTE
        7: TURN-LEFT ROBOT1 NORTE OESTE
        8: GOTO ROBOT1 OESTE LOC7 LOC6
        9: TURN-RIGHT ROBOT1 OESTE NORTE
       10: GOTO ROBOT1 NORTE LOC6 LOC1
       11: GIVE ROBOT1 LOC1 OSCAR1 PRINCESA1
       12: TURN-RIGHT ROBOT1 NORTE ESTE
       13: GOTO ROBOT1 ESTE LOC1 LOC2
       14: TURN-RIGHT ROBOT1 ESTE SUR
       15: GOTO ROBOT1 SUR LOC2 LOC7
       16: GOTO ROBOT1 SUR LOC7 LOC12
       17: GOTO ROBOT1 SUR LOC12 LOC17
       18: PICK ROBOT1 LOC17 ROSA1
       19: GOTO ROBOT1 SUR LOC17 LOC22
       20: TURN-RIGHT ROBOT1 SUR OESTE
       21: GOTO ROBOT1 OESTE LOC22 LOC21
       22: GIVE ROBOT1 LOC21 ROSA1 BRUJA1
       23: TURN-RIGHT ROBOT1 OESTE NORTE
       24: TURN-RIGHT ROBOT1 NORTE ESTE
       25: GOTO ROBOT1 ESTE LOC21 LOC22
       26: TURN-LEFT ROBOT1 ESTE NORTE
       27: GOTO ROBOT1 NORTE LOC22 LOC17
       28: TURN-RIGHT ROBOT1 NORTE ESTE
       29: GOTO ROBOT1 ESTE LOC17 LOC18
       30: TURN-LEFT ROBOT1 ESTE NORTE
       31: GOTO ROBOT1 NORTE LOC18 LOC13
       32: TURN-LEFT ROBOT1 NORTE OESTE
       33: PICK ROBOT1 LOC13 ORO1
       34: GIVE ROBOT1 LOC13 ROSA1 LEONARDO1
       35: TURN-RIGHT ROBOT1 OESTE NORTE
       36: TURN-RIGHT ROBOT1 NORTE ESTE
       37: GOTO ROBOT1 ESTE LOC13 LOC14
       38: TURN-RIGHT ROBOT1 ESTE SUR
       39: GOTO ROBOT1 SUR LOC14 LOC19
       40: PICK ROBOT1 LOC19 ALGORITMO1
       41: TURN-LEFT ROBOT1 SUR ESTE
       42: GOTO ROBOT1 ESTE LOC19 LOC20
       43: TURN-RIGHT ROBOT1 ESTE SUR
       44: GOTO ROBOT1 SUR LOC20 LOC25
       45: DROP ROBOT1 LOC25 OSCAR1
       46: TURN-LEFT ROBOT1 SUR ESTE
       47: TURN-LEFT ROBOT1 ESTE NORTE
       48: PICK ROBOT1 LOC25 OSCAR1
       49: GIVE ROBOT1 LOC25 ORO1 PROFESOR1
       50: GOTO ROBOT1 NORTE LOC25 LOC20
       51: GOTO ROBOT1 NORTE LOC20 LOC15
       52: GOTO ROBOT1 NORTE LOC15 LOC10
       53: GOTO ROBOT1 NORTE LOC10 LOC5
       54: TURN-LEFT ROBOT1 NORTE OESTE
       55: GOTO ROBOT1 OESTE LOC5 LOC4
       56: TURN-LEFT ROBOT1 OESTE SUR
       57: GOTO ROBOT1 SUR LOC4 LOC9
       58: PICK ROBOT1 LOC9 MANZANA1
       59: DROP ROBOT1 LOC9 MANZANA1
       60: TURN-LEFT ROBOT1 SUR ESTE
       61: TURN-LEFT ROBOT1 ESTE NORTE
       62: PICK ROBOT1 LOC9 MANZANA1
       63: GOTO ROBOT1 NORTE LOC9 LOC4
       64: TURN-RIGHT ROBOT1 NORTE ESTE
       65: GOTO ROBOT1 ESTE LOC4 LOC5
       66: DROP ROBOT1 LOC5 OSCAR1
       67: TURN-LEFT ROBOT1 ESTE NORTE
       68: TURN-LEFT ROBOT1 NORTE OESTE
       69: PICK ROBOT1 LOC5 OSCAR1
       70: GIVE ROBOT1 LOC5 ALGORITMO1 PRINCIPE1


time spent:    0.00 seconds instantiating 945 easy, 0 hard action templates
               0.00 seconds reachability analysis, yielding 171 facts and 345 actions
               0.00 seconds creating final representation with 166 relevant facts, 1 relevant fluents
               0.00 seconds computing LNF
               0.00 seconds building connectivity graph
               1.72 seconds searching, evaluating 51090 states, to a max depth of 0
               1.72 seconds total time
\end{lstlisting}
\subsection*{2.2}
\begin{lstlisting}
ff: parsing domain file
domain 'BELKAN' defined
 ... done.
ff: parsing problem file
problem 'EJ1' defined
 ... done.


metric established (normalized to minimize): ((1.00*[RF0](PATHLENGTH_ROBOT1)) - () + 0.00)

checking for cyclic := effects --- OK.

ff: search configuration is  best-first on 1*g(s) + 2*h(s) where
    metric is ((1.00*[RF0](PATHLENGTH_ROBOT1)) - () + 0.00)

advancing to distance:   23
                         22
                         21
                         19
                         18
                         17
                         16
                         15
                         14
                         13
                         12
                         11
                         10
                          9
                          8
                          7
                          6
                          5
                          4
                          3
                          2
                          1
                          0

ff: found legal plan as follows

step    0: TURN-RIGHT ROBOT1 NORTE ESTE
        1: GOTO ROBOT1 ESTE LOC1 LOC2
        2: TURN-RIGHT ROBOT1 ESTE SUR
        3: GOTO ROBOT1 SUR LOC2 LOC7
        4: TURN-LEFT ROBOT1 SUR ESTE
        5: TURN-LEFT ROBOT1 ESTE NORTE
        6: PICK ROBOT1 LOC7 OSCAR1
        7: TURN-LEFT ROBOT1 NORTE OESTE
        8: GOTO ROBOT1 OESTE LOC7 LOC6
        9: TURN-RIGHT ROBOT1 OESTE NORTE
       10: GOTO ROBOT1 NORTE LOC6 LOC1
       11: GIVE ROBOT1 LOC1 OSCAR1 PRINCESA1
       12: TURN-RIGHT ROBOT1 NORTE ESTE
       13: GOTO ROBOT1 ESTE LOC1 LOC2
       14: TURN-RIGHT ROBOT1 ESTE SUR
       15: GOTO ROBOT1 SUR LOC2 LOC7
       16: GOTO ROBOT1 SUR LOC7 LOC12
       17: GOTO ROBOT1 SUR LOC12 LOC17
       18: PICK ROBOT1 LOC17 ROSA1
       19: GOTO ROBOT1 SUR LOC17 LOC22
       20: TURN-RIGHT ROBOT1 SUR OESTE
       21: GOTO ROBOT1 OESTE LOC22 LOC21
       22: GIVE ROBOT1 LOC21 ROSA1 BRUJA1
       23: TURN-RIGHT ROBOT1 OESTE NORTE
       24: TURN-RIGHT ROBOT1 NORTE ESTE
       25: GOTO ROBOT1 ESTE LOC21 LOC22
       26: TURN-LEFT ROBOT1 ESTE NORTE
       27: GOTO ROBOT1 NORTE LOC22 LOC17
       28: TURN-RIGHT ROBOT1 NORTE ESTE
       29: GOTO ROBOT1 ESTE LOC17 LOC18
       30: TURN-LEFT ROBOT1 ESTE NORTE
       31: GOTO ROBOT1 NORTE LOC18 LOC13
       32: TURN-LEFT ROBOT1 NORTE OESTE
       33: PICK ROBOT1 LOC13 ORO1
       34: GIVE ROBOT1 LOC13 ROSA1 LEONARDO1
       35: TURN-RIGHT ROBOT1 OESTE NORTE
       36: TURN-RIGHT ROBOT1 NORTE ESTE
       37: GOTO ROBOT1 ESTE LOC13 LOC14
       38: TURN-RIGHT ROBOT1 ESTE SUR
       39: GOTO ROBOT1 SUR LOC14 LOC19
       40: PICK ROBOT1 LOC19 ALGORITMO1
       41: TURN-LEFT ROBOT1 SUR ESTE
       42: GOTO ROBOT1 ESTE LOC19 LOC20
       43: TURN-RIGHT ROBOT1 ESTE SUR
       44: GOTO ROBOT1 SUR LOC20 LOC25
       45: DROP ROBOT1 LOC25 OSCAR1
       46: TURN-LEFT ROBOT1 SUR ESTE
       47: TURN-LEFT ROBOT1 ESTE NORTE
       48: PICK ROBOT1 LOC25 OSCAR1
       49: GIVE ROBOT1 LOC25 ORO1 PROFESOR1
       50: GOTO ROBOT1 NORTE LOC25 LOC20
       51: GOTO ROBOT1 NORTE LOC20 LOC15
       52: GOTO ROBOT1 NORTE LOC15 LOC10
       53: GOTO ROBOT1 NORTE LOC10 LOC5
       54: TURN-LEFT ROBOT1 NORTE OESTE
       55: GOTO ROBOT1 OESTE LOC5 LOC4
       56: TURN-LEFT ROBOT1 OESTE SUR
       57: GOTO ROBOT1 SUR LOC4 LOC9
       58: PICK ROBOT1 LOC9 MANZANA1
       59: DROP ROBOT1 LOC9 MANZANA1
       60: TURN-LEFT ROBOT1 SUR ESTE
       61: TURN-LEFT ROBOT1 ESTE NORTE
       62: PICK ROBOT1 LOC9 MANZANA1
       63: GOTO ROBOT1 NORTE LOC9 LOC4
       64: TURN-RIGHT ROBOT1 NORTE ESTE
       65: GOTO ROBOT1 ESTE LOC4 LOC5
       66: DROP ROBOT1 LOC5 OSCAR1
       67: TURN-LEFT ROBOT1 ESTE NORTE
       68: TURN-LEFT ROBOT1 NORTE OESTE
       69: PICK ROBOT1 LOC5 OSCAR1
       70: GIVE ROBOT1 LOC5 ALGORITMO1 PRINCIPE1


time spent:    0.00 seconds instantiating 945 easy, 0 hard action templates
               0.00 seconds reachability analysis, yielding 171 facts and 345 actions
               0.00 seconds creating final representation with 166 relevant facts, 1 relevant fluents
               0.00 seconds computing LNF
               0.00 seconds building connectivity graph
               1.66 seconds searching, evaluating 51516 states, to a max depth of 0
               1.66 seconds total time
\end{lstlisting}
\subsection*{3.1}
\begin{lstlisting}
ff: parsing domain file
domain 'BELKAN' defined
 ... done.
ff: parsing problem file
problem 'EJ1' defined
 ... done.


no metric specified. plan length assumed.

checking for cyclic := effects --- OK.

ff: search configuration is  best-first on 1*g(s) + 2*h(s) where
    metric is  plan length

advancing to distance:   17
                         16
                         15
                         14
                         12
                         11
                         10
                          9
                          8
                          7
                          6
                          5
                          4
                          3
                          2
                          1
                          0

ff: found legal plan as follows

step    0: TURN-RIGHT ROBOT1 NORTE ESTE
        1: GOTO ROBOT1 ESTE LOC1 LOC2
        2: TURN-RIGHT ROBOT1 ESTE SUR
        3: GOTO ROBOT1 SUR LOC2 LOC7
        4: GOTO ROBOT1 SUR LOC7 LOC12
        5: GOTO ROBOT1 SUR LOC12 LOC17
        6: TURN-RIGHT ROBOT1 SUR OESTE
        7: GOTO ROBOT1 OESTE LOC17 LOC16
        8: TURN-LEFT ROBOT1 OESTE SUR
        9: TURN-LEFT ROBOT1 SUR ESTE
       10: PICK ROBOT1 LOC16 BIKINI
       11: GOTO ROBOT1 ESTE LOC16 LOC17
       12: PUSH-BAG ROBOT1 BIKINI
       13: TURN-RIGHT ROBOT1 ESTE SUR
       14: GOTO-SPECIAL ROBOT1 SUR LOC17 LOC22 AGUA BIKINI
       15: PICK ROBOT1 LOC22 ZAPATILLA
       16: TURN-RIGHT ROBOT1 SUR OESTE
       17: GOTO ROBOT1 OESTE LOC22 LOC21
       18: TURN-RIGHT ROBOT1 OESTE NORTE
       19: DROP ROBOT1 LOC21 ZAPATILLA
       20: PULL-BAG ROBOT1 BIKINI
       21: DROP ROBOT1 LOC21 BIKINI
       22: TURN-RIGHT ROBOT1 NORTE ESTE
       23: PICK ROBOT1 LOC21 ZAPATILLA
       24: PUSH-BAG ROBOT1 ZAPATILLA
       25: PICK ROBOT1 LOC21 BIKINI
       26: GOTO-SPECIAL ROBOT1 ESTE LOC21 LOC22 AGUA BIKINI
       27: GOTO-SPECIAL ROBOT1 ESTE LOC22 LOC23 BOSQUE ZAPATILLA
       28: DROP-SPECIAL ROBOT1 LOC23 BIKINI BOSQUE
       29: PICK ROBOT1 LOC23 ALGORITMO
       30: GOTO ROBOT1 ESTE LOC23 LOC24
       31: GOTO ROBOT1 ESTE LOC24 LOC25
       32: GIVE ROBOT1 LOC25 ALGORITMO PROFESOR


time spent:    0.00 seconds instantiating 1288 easy, 0 hard action templates
               0.00 seconds reachability analysis, yielding 340 facts and 1175 actions
               0.00 seconds creating final representation with 340 relevant facts, 1 relevant fluents
               0.00 seconds computing LNF
               0.00 seconds building connectivity graph
               1.05 seconds searching, evaluating 14709 states, to a max depth of 0
               1.05 seconds total time
\end{lstlisting}
\subsection*{4.1}
\begin{lstlisting}
ff: parsing domain file
domain 'BELKAN' defined
 ... done.
ff: parsing problem file
problem 'EJ1' defined
 ... done.


no metric specified. plan length assumed.

checking for cyclic := effects --- OK.

ff: search configuration is  best-first on 1*g(s) + 5*h(s) where
    metric is  plan length

advancing to distance:   30
                         29
                         28
                         27
                         24
                         23
                         22
                         21
                         20
                         19
                         18
                         17
                         16
                         15
                         14
                         13
                         11
                         10
                          9
                          5
                          4
                          3
                          2
                          1
                          0

ff: found legal plan as follows

step    0: TURN-RIGHT ROBOT1 NORTE ESTE
        1: GOTO ROBOT1 ESTE LOC1 LOC2
        2: TURN-RIGHT ROBOT1 ESTE SUR
        3: GOTO ROBOT1 SUR LOC2 LOC7
        4: GOTO ROBOT1 SUR LOC7 LOC12
        5: TURN-RIGHT ROBOT1 SUR OESTE
        6: GOTO ROBOT1 OESTE LOC12 LOC11
        7: TURN-LEFT ROBOT1 OESTE SUR
        8: GOTO ROBOT1 SUR LOC11 LOC16
        9: PICK ROBOT1 LOC16 BIKINI
       10: PUSH-BAG ROBOT1 BIKINI
       11: TURN-LEFT ROBOT1 SUR ESTE
       12: GOTO ROBOT1 ESTE LOC16 LOC17
       13: TURN-RIGHT ROBOT1 ESTE SUR
       14: PICK ROBOT1 LOC17 MANZANA
       15: GOTO-SPECIAL ROBOT1 SUR LOC17 LOC22 AGUA BIKINI
       16: TURN-RIGHT ROBOT1 SUR OESTE
       17: GOTO ROBOT1 OESTE LOC22 LOC21
       18: TURN-RIGHT ROBOT1 OESTE NORTE
       19: TURN-RIGHT ROBOT1 NORTE ESTE
       20: GIVE ROBOT1 LOC21 MANZANA BRUJA
       21: GOTO-SPECIAL ROBOT1 ESTE LOC21 LOC22 AGUA BIKINI
       22: TURN-LEFT ROBOT1 ESTE NORTE
       23: GOTO ROBOT1 NORTE LOC22 LOC17
       24: GOTO ROBOT1 NORTE LOC17 LOC12
       25: GOTO ROBOT1 NORTE LOC12 LOC7
       26: PICK ROBOT1 LOC7 OSCAR
       27: TURN-RIGHT ROBOT1 NORTE ESTE
       28: TURN-RIGHT ROBOT1 ESTE SUR
       29: GOTO ROBOT1 SUR LOC7 LOC12
       30: TURN-LEFT ROBOT1 SUR ESTE
       31: GOTO ROBOT1 ESTE LOC12 LOC13
       32: GIVE ROBOT1 LOC13 OSCAR LEONARDO
       33: GOTO ROBOT1 ESTE LOC13 LOC14
       34: PICK ROBOT1 LOC14 OSCAR
       35: TURN-RIGHT ROBOT1 ESTE SUR
       36: TURN-RIGHT ROBOT1 SUR OESTE
       37: GOTO ROBOT1 OESTE LOC14 LOC13
       38: GIVE ROBOT1 LOC13 OSCAR LEONARDO
       39: PICK ROBOT1 LOC13 MANZANA
       40: GOTO ROBOT1 OESTE LOC13 LOC12
       41: TURN-RIGHT ROBOT1 OESTE NORTE
       42: GOTO ROBOT1 NORTE LOC12 LOC7
       43: DROP ROBOT1 LOC7 MANZANA
       44: TURN-RIGHT ROBOT1 NORTE ESTE
       45: GOTO ROBOT1 ESTE LOC7 LOC8
       46: GOTO ROBOT1 ESTE LOC8 LOC9
       47: PICK ROBOT1 LOC9 OSCAR
       48: TURN-LEFT ROBOT1 ESTE NORTE
       49: TURN-LEFT ROBOT1 NORTE OESTE
       50: GOTO ROBOT1 OESTE LOC9 LOC8
       51: GOTO ROBOT1 OESTE LOC8 LOC7
       52: GOTO ROBOT1 OESTE LOC7 LOC6
       53: TURN-RIGHT ROBOT1 OESTE NORTE
       54: GOTO ROBOT1 NORTE LOC6 LOC1
       55: TURN-RIGHT ROBOT1 NORTE ESTE
       56: GIVE ROBOT1 LOC1 OSCAR PRINCESA
       57: GOTO ROBOT1 ESTE LOC1 LOC2
       58: TURN-RIGHT ROBOT1 ESTE SUR
       59: GOTO ROBOT1 SUR LOC2 LOC7
       60: TURN-LEFT ROBOT1 SUR ESTE
       61: PICK ROBOT1 LOC7 MANZANA
       62: GOTO ROBOT1 ESTE LOC7 LOC8
       63: TURN-LEFT ROBOT1 ESTE NORTE
       64: GOTO ROBOT1 NORTE LOC8 LOC3
       65: TURN-RIGHT ROBOT1 NORTE ESTE
       66: GOTO ROBOT1 ESTE LOC3 LOC4
       67: GOTO ROBOT1 ESTE LOC4 LOC5
       68: TURN-RIGHT ROBOT1 ESTE SUR
       69: GIVE ROBOT1 LOC5 MANZANA PRINCIPE
       70: GOTO ROBOT1 SUR LOC5 LOC10
       71: GOTO ROBOT1 SUR LOC10 LOC15
       72: TURN-RIGHT ROBOT1 SUR OESTE
       73: GOTO ROBOT1 OESTE LOC15 LOC14
       74: GOTO ROBOT1 OESTE LOC14 LOC13
       75: GOTO ROBOT1 OESTE LOC13 LOC12
       76: TURN-LEFT ROBOT1 OESTE SUR
       77: GOTO ROBOT1 SUR LOC12 LOC17
       78: GOTO-SPECIAL ROBOT1 SUR LOC17 LOC22 AGUA BIKINI
       79: PICK ROBOT1 LOC22 ZAPATILLA
       80: TURN-LEFT ROBOT1 SUR ESTE
       81: GOTO-SPECIAL ROBOT1 ESTE LOC22 LOC23 BOSQUE ZAPATILLA
       82: GOTO ROBOT1 ESTE LOC23 LOC24
       83: GOTO ROBOT1 ESTE LOC24 LOC25
       84: TURN-LEFT ROBOT1 ESTE NORTE
       85: TURN-LEFT ROBOT1 NORTE OESTE
       86: DROP ROBOT1 LOC25 ZAPATILLA
       87: PULL-BAG ROBOT1 BIKINI
       88: DROP ROBOT1 LOC25 BIKINI
       89: PICK ROBOT1 LOC25 ZAPATILLA
       90: GOTO ROBOT1 OESTE LOC25 LOC24
       91: GOTO-SPECIAL ROBOT1 OESTE LOC24 LOC23 BOSQUE ZAPATILLA
       92: PUSH-BAG ROBOT1 ZAPATILLA
       93: PICK ROBOT1 LOC23 MANZANA
       94: TURN-RIGHT ROBOT1 OESTE NORTE
       95: TURN-RIGHT ROBOT1 NORTE ESTE
       96: GOTO ROBOT1 ESTE LOC23 LOC24
       97: GOTO ROBOT1 ESTE LOC24 LOC25
       98: GIVE ROBOT1 LOC25 MANZANA PROFESOR


time spent:    0.00 seconds instantiating 1043 easy, 0 hard action templates
               0.00 seconds reachability analysis, yielding 148 facts and 267 actions
               0.00 seconds creating final representation with 140 relevant facts, 2 relevant fluents
               0.00 seconds computing LNF
               0.00 seconds building connectivity graph
              31.46 seconds searching, evaluating 142745 states, to a max depth of 0
              31.46 seconds total time
\end{lstlisting}
\subsection*{5.1}
\begin{lstlisting}
ff: parsing domain file
domain 'BELKAN' defined
 ... done.
ff: parsing problem file
problem 'EJ1' defined
 ... done.


no metric specified. plan length assumed.

checking for cyclic := effects --- OK.

ff: search configuration is  best-first on 1*g(s) + 5*h(s) where
    metric is  plan length

advancing to distance:   25
                         24
                         22
                         21
                         20
                         19
                         18
                         17
                         16
                         14
                         13
                         12
                         11
                         10
                          9
                          8
                          7
                          6
                          5
                          4
                          3
                          2
                          1
                          0

ff: found legal plan as follows

step    0: TURN-RIGHT ROBOT1 NORTE ESTE
        1: GOTO ROBOT1 ESTE LOC1 LOC2
        2: TURN-RIGHT ROBOT1 ESTE SUR
        3: GOTO ROBOT1 SUR LOC2 LOC7
        4: GOTO ROBOT1 SUR LOC7 LOC12
        5: TURN-RIGHT ROBOT1 SUR OESTE
        6: GOTO ROBOT1 OESTE LOC12 LOC11
        7: TURN-LEFT ROBOT1 OESTE SUR
        8: GOTO ROBOT1 SUR LOC11 LOC16
        9: PICK ROBOT1 LOC16 BIKINI
       10: PUSH-BAG ROBOT1 BIKINI
       11: TURN-LEFT ROBOT1 SUR ESTE
       12: GOTO ROBOT1 ESTE LOC16 LOC17
       13: TURN-RIGHT ROBOT1 ESTE SUR
       14: PICK ROBOT1 LOC17 MANZANA
       15: GOTO-SPECIAL ROBOT1 SUR LOC17 LOC22 AGUA BIKINI
       16: TURN-RIGHT ROBOT1 SUR OESTE
       17: GOTO ROBOT1 OESTE LOC22 LOC21
       18: TURN-RIGHT ROBOT1 OESTE NORTE
       19: TURN-RIGHT ROBOT1 NORTE ESTE
       20: GIVE ROBOT1 LOC21 MANZANA BRUJA
       21: GOTO-SPECIAL ROBOT1 ESTE LOC21 LOC22 AGUA BIKINI
       22: TURN-LEFT ROBOT1 ESTE NORTE
       23: GOTO ROBOT1 NORTE LOC22 LOC17
       24: GOTO ROBOT1 NORTE LOC17 LOC12
       25: GOTO ROBOT1 NORTE LOC12 LOC7
       26: PICK ROBOT1 LOC7 OSCAR
       27: GOTO ROBOT1 NORTE LOC7 LOC2
       28: TURN-LEFT ROBOT1 NORTE OESTE
       29: GOTO ROBOT1 OESTE LOC2 LOC1
       30: TURN-LEFT ROBOT1 OESTE SUR
       31: GIVE ROBOT1 LOC1 OSCAR PRINCESA
       32: GOTO ROBOT1 SUR LOC1 LOC6
       33: TURN-LEFT ROBOT1 SUR ESTE
       34: GOTO ROBOT1 ESTE LOC6 LOC7
       35: GOTO ROBOT1 ESTE LOC7 LOC8
       36: GOTO ROBOT1 ESTE LOC8 LOC9
       37: TURN-LEFT ROBOT1 ESTE NORTE
       38: TURN-LEFT ROBOT1 NORTE OESTE
       39: PICK ROBOT1 LOC9 OSCAR
       40: GOTO ROBOT1 OESTE LOC9 LOC8
       41: GOTO ROBOT1 OESTE LOC8 LOC7
       42: TURN-LEFT ROBOT1 OESTE SUR
       43: DROP ROBOT1 LOC7 OSCAR
       44: GOTO ROBOT1 SUR LOC7 LOC12
       45: TURN-LEFT ROBOT1 SUR ESTE
       46: GOTO ROBOT1 ESTE LOC12 LOC13
       47: GOTO ROBOT1 ESTE LOC13 LOC14
       48: TURN-RIGHT ROBOT1 ESTE SUR
       49: TURN-RIGHT ROBOT1 SUR OESTE
       50: PICK ROBOT1 LOC14 OSCAR
       51: GOTO ROBOT1 OESTE LOC14 LOC13
       52: TURN-LEFT ROBOT1 OESTE SUR
       53: GIVE ROBOT1 LOC13 OSCAR LEONARDO
       54: PICK ROBOT1 LOC13 MANZANA
       55: GOTO ROBOT1 SUR LOC13 LOC18
       56: TURN-RIGHT ROBOT1 SUR OESTE
       57: GOTO ROBOT1 OESTE LOC18 LOC17
       58: TURN-LEFT ROBOT1 OESTE SUR
       59: GOTO-SPECIAL ROBOT1 SUR LOC17 LOC22 AGUA BIKINI
       60: TURN-RIGHT ROBOT1 SUR OESTE
       61: GOTO ROBOT1 OESTE LOC22 LOC21
       62: TURN-RIGHT ROBOT1 OESTE NORTE
       63: TURN-RIGHT ROBOT1 NORTE ESTE
       64: GIVE ROBOT1 LOC21 MANZANA BRUJA
       65: GOTO-SPECIAL ROBOT1 ESTE LOC21 LOC22 AGUA BIKINI
       66: PICK ROBOT1 LOC22 ZAPATILLA
       67: GOTO-SPECIAL ROBOT1 ESTE LOC22 LOC23 BOSQUE ZAPATILLA
       68: GOTO ROBOT1 ESTE LOC23 LOC24
       69: GOTO ROBOT1 ESTE LOC24 LOC25
       70: TURN-LEFT ROBOT1 ESTE NORTE
       71: TURN-LEFT ROBOT1 NORTE OESTE
       72: DROP ROBOT1 LOC25 ZAPATILLA
       73: PULL-BAG ROBOT1 BIKINI
       74: DROP ROBOT1 LOC25 BIKINI
       75: PICK ROBOT1 LOC25 ZAPATILLA
       76: GOTO ROBOT1 OESTE LOC25 LOC24
       77: GOTO-SPECIAL ROBOT1 OESTE LOC24 LOC23 BOSQUE ZAPATILLA
       78: PUSH-BAG ROBOT1 ZAPATILLA
       79: PICK ROBOT1 LOC23 MANZANA
       80: TURN-RIGHT ROBOT1 OESTE NORTE
       81: TURN-RIGHT ROBOT1 NORTE ESTE
       82: GOTO ROBOT1 ESTE LOC23 LOC24
       83: GOTO ROBOT1 ESTE LOC24 LOC25
       84: TURN-LEFT ROBOT1 ESTE NORTE
       85: GIVE ROBOT1 LOC25 MANZANA PROFESOR
       86: PICK ROBOT1 LOC25 BIKINI
       87: TURN-LEFT ROBOT1 NORTE OESTE
       88: GOTO ROBOT1 OESTE LOC25 LOC24
       89: GOTO-SPECIAL ROBOT1 OESTE LOC24 LOC23 BOSQUE ZAPATILLA
       90: GOTO-SPECIAL ROBOT1 OESTE LOC23 LOC22 AGUA BIKINI
       91: TURN-RIGHT ROBOT1 OESTE NORTE
       92: GOTO ROBOT1 NORTE LOC22 LOC17
       93: GOTO ROBOT1 NORTE LOC17 LOC12
       94: GOTO ROBOT1 NORTE LOC12 LOC7
       95: DROP-SPECIAL ROBOT1 LOC7 BIKINI ARENA
       96: PICK ROBOT1 LOC7 OSCAR
       97: GOTO ROBOT1 NORTE LOC7 LOC2
       98: TURN-LEFT ROBOT1 NORTE OESTE
       99: GOTO ROBOT1 OESTE LOC2 LOC1
      100: GIVE ROBOT1 LOC1 OSCAR PRINCESA


time spent:    0.00 seconds instantiating 1043 easy, 0 hard action templates
               0.00 seconds reachability analysis, yielding 148 facts and 267 actions
               0.00 seconds creating final representation with 140 relevant facts, 7 relevant fluents
               0.00 seconds computing LNF
               0.00 seconds building connectivity graph
              32.36 seconds searching, evaluating 186591 states, to a max depth of 0
              32.36 seconds total time
\end{lstlisting}
\subsection*{6.1}
\begin{lstlisting}
ff: parsing domain file
domain 'BELKAN' defined
 ... done.
ff: parsing problem file
problem 'EJ1' defined
 ... done.


no metric specified. plan length assumed.

checking for cyclic := effects --- OK.

ff: search configuration is EHC, if that fails then  best-first on 1*g(s) + 5*h(s) where
    metric is  plan length

Cueing down from goal distance:   23 into depth [1]
                                  22            [1]
                                  21            [1][2][3]
                                  20            [1][2][3][4][5]
                                  19            [1][2][3]
                                  18            [1][2]
                                  17            [1]
                                  16            [1][2]
                                  15            [1][2]
                                  14            [1][2]
                                  13            [1][2][3]
                                  12            [1]
                                  11            [1][2]
                                  10            [1]
                                   9            [1][2]
                                   8            [1]
                                   7            [1][2][3][4][5][6]
                                   6            [1]
                                   5            [1]
                                   4            [1]
                                   3            [1]
                                   2            [1][2][3][4][5][6][7][8][9][10][11][12][13][14][15][16][17][18][19][20]
                                   1            [1]
                                   0

ff: found legal plan as follows

step    0: TURN-LEFT ROBOT2 NORTE OESTE
        1: TURN-LEFT ROBOT1 NORTE OESTE
        2: TURN-LEFT ROBOT1 OESTE SUR
        3: GOTO ROBOT1 SUR LOC1 LOC6
        4: TURN-LEFT ROBOT1 SUR ESTE
        5: GOTO ROBOT1 ESTE LOC6 LOC7
        6: TURN-RIGHT ROBOT1 ESTE SUR
        7: TURN-RIGHT ROBOT1 SUR OESTE
        8: PICK ROBOT1 LOC7 OSCAR
        9: TURN-LEFT ROBOT1 OESTE SUR
       10: GOTO ROBOT1 SUR LOC7 LOC12
       11: TURN-LEFT ROBOT1 SUR ESTE
       12: GOTO ROBOT1 ESTE LOC12 LOC13
       13: GIVE ROBOT1 LOC13 OSCAR LEONARDO
       14: TURN-RIGHT ROBOT1 ESTE SUR
       15: GOTO ROBOT1 SUR LOC13 LOC18
       16: TURN-RIGHT ROBOT1 SUR OESTE
       17: GOTO ROBOT1 OESTE LOC18 LOC17
       18: GOTO ROBOT1 OESTE LOC17 LOC16
       19: TURN-LEFT ROBOT1 OESTE SUR
       20: PICK ROBOT1 LOC16 BIKINI
       21: TURN-LEFT ROBOT1 SUR ESTE
       22: GOTO ROBOT1 ESTE LOC16 LOC17
       23: PUSH-BAG ROBOT1 BIKINI
       24: TURN-RIGHT ROBOT1 ESTE SUR
       25: GOTO-SPECIAL ROBOT1 SUR LOC17 LOC22 AGUA BIKINI
       26: TURN-LEFT ROBOT1 SUR ESTE
       27: PICK ROBOT1 LOC22 ZAPATILLA
       28: GOTO-SPECIAL ROBOT1 ESTE LOC22 LOC23 BOSQUE ZAPATILLA
       29: GOTO ROBOT1 ESTE LOC23 LOC24
       30: DROP ROBOT1 LOC24 ZAPATILLA
       31: GOTO ROBOT2 OESTE LOC25 LOC24
       32: PICK ROBOT2 LOC24 ZAPATILLA
       33: PUSH-BAG ROBOT2 ZAPATILLA
       34: GOTO-SPECIAL ROBOT2 OESTE LOC24 LOC23 BOSQUE ZAPATILLA
       35: TURN-LEFT ROBOT2 OESTE SUR
       36: TURN-LEFT ROBOT2 SUR ESTE
       37: PICK ROBOT2 LOC23 MANZANA
       38: GOTO ROBOT2 ESTE LOC23 LOC24
       39: GOTO ROBOT2 ESTE LOC24 LOC25
       40: GIVE ROBOT2 LOC25 MANZANA PROFESOR
       41: PULL-BAG ROBOT1 BIKINI
       42: TURN-RIGHT ROBOT2 ESTE SUR
       43: TURN-RIGHT ROBOT2 SUR OESTE
       44: GOTO ROBOT2 OESTE LOC25 LOC24
       45: DROP ROBOT1 LOC24 BIKINI
       46: PICK ROBOT2 LOC24 BIKINI
       47: GOTO-SPECIAL ROBOT2 OESTE LOC24 LOC23 BOSQUE ZAPATILLA
       48: GOTO-SPECIAL ROBOT2 OESTE LOC23 LOC22 AGUA BIKINI
       49: TURN-RIGHT ROBOT2 OESTE NORTE
       50: GOTO ROBOT2 NORTE LOC22 LOC17
       51: GOTO ROBOT2 NORTE LOC17 LOC12
       52: TURN-RIGHT ROBOT2 NORTE ESTE
       53: DROP ROBOT2 LOC12 BIKINI
       54: GOTO ROBOT2 ESTE LOC12 LOC13
       55: GOTO ROBOT2 ESTE LOC13 LOC14
       56: TURN-RIGHT ROBOT2 ESTE SUR
       57: TURN-RIGHT ROBOT2 SUR OESTE
       58: PICK ROBOT2 LOC14 OSCAR
       59: GOTO ROBOT2 OESTE LOC14 LOC13
       60: GIVE ROBOT2 LOC13 OSCAR LEONARDO


time spent:    0.00 seconds instantiating 2086 easy, 0 hard action templates
               0.00 seconds reachability analysis, yielding 190 facts and 534 actions
               0.00 seconds creating final representation with 179 relevant facts, 10 relevant fluents
               0.00 seconds computing LNF
               0.00 seconds building connectivity graph
               0.04 seconds searching, evaluating 1835 states, to a max depth of 20
               0.04 seconds total time
\end{lstlisting}
\subsection*{6.2}
\begin{lstlisting}
in.pddl -f ej6_problem.2.pddl

ff: parsing domain file
domain 'BELKAN' defined
 ... done.
ff: parsing problem file
problem 'EJ1' defined
 ... done.


no metric specified. plan length assumed.

checking for cyclic := effects --- OK.

ff: search configuration is EHC, if that fails then  best-first on 1*g(s) + 5*h(s) where
    metric is  plan length

Cueing down from goal distance:   10 into depth [1]
                                   9            [1]
                                   8            [1][2][3][4]
                                   6            [1]
                                   5            [1]
                                   4            [1]
                                   3            [1]
                                   2            [1][2][3][4][5][6]
                                   1            [1] --- pruning stopped --- [1][2][3][4][5][6][7][8][9][10][11][12][13][14][15][16][17][18][19][20][21][22][23][24][25][26][27]
                                   0            

ff: found legal plan as follows

step    0: TURN-LEFT ROBOT1 NORTE OESTE
        1: TURN-LEFT ROBOT1 OESTE SUR
        2: GOTO ROBOT1 SUR LOC1 LOC6
        3: TURN-LEFT ROBOT1 SUR ESTE
        4: GOTO ROBOT1 ESTE LOC6 LOC7
        5: TURN-RIGHT ROBOT1 ESTE SUR
        6: PICK ROBOT1 LOC7 OSCAR
        7: GOTO ROBOT1 SUR LOC7 LOC12
        8: TURN-LEFT ROBOT1 SUR ESTE
        9: GOTO ROBOT1 ESTE LOC12 LOC13
       10: GIVE ROBOT1 LOC13 OSCAR LEONARDO
       11: GOTO ROBOT1 ESTE LOC13 LOC14
       12: TURN-RIGHT ROBOT1 ESTE SUR
       13: TURN-RIGHT ROBOT1 SUR OESTE
       14: PICK ROBOT1 LOC14 OSCAR
       15: GOTO ROBOT1 OESTE LOC14 LOC13
       16: TURN-LEFT ROBOT2 NORTE OESTE
       17: GOTO ROBOT1 OESTE LOC13 LOC12
       18: GOTO ROBOT2 OESTE LOC25 LOC24
       19: GOTO ROBOT1 OESTE LOC12 LOC11
       20: TURN-LEFT ROBOT1 OESTE SUR
       21: DROP ROBOT1 LOC11 OSCAR
       22: GOTO ROBOT1 SUR LOC11 LOC16
       23: TURN-LEFT ROBOT1 SUR ESTE
       24: PICK ROBOT1 LOC16 BIKINI
       25: GOTO ROBOT1 ESTE LOC16 LOC17
       26: TURN-RIGHT ROBOT1 ESTE SUR
       27: PUSH-BAG ROBOT1 BIKINI
       28: GOTO-SPECIAL ROBOT1 SUR LOC17 LOC22 AGUA BIKINI
       29: TURN-LEFT ROBOT1 SUR ESTE
       30: PICK ROBOT1 LOC22 ZAPATILLA
       31: GOTO-SPECIAL ROBOT1 ESTE LOC22 LOC23 BOSQUE ZAPATILLA
       32: GOTO ROBOT1 ESTE LOC23 LOC24
       33: DROP ROBOT1 LOC24 ZAPATILLA
       34: PICK ROBOT2 LOC24 ZAPATILLA
       35: GOTO-SPECIAL ROBOT2 OESTE LOC24 LOC23 BOSQUE ZAPATILLA
       36: TURN-LEFT ROBOT2 OESTE SUR
       37: PUSH-BAG ROBOT2 ZAPATILLA
       38: TURN-LEFT ROBOT2 SUR ESTE
       39: PICK ROBOT2 LOC23 MANZANA
       40: GOTO ROBOT2 ESTE LOC23 LOC24
       41: GOTO ROBOT2 ESTE LOC24 LOC25
       42: GIVE ROBOT2 LOC25 MANZANA PROFESOR
     

time spent:    0.00 seconds instantiating 2086 easy, 0 hard action templates
               0.00 seconds reachability analysis, yielding 190 facts and 534 actions
               0.00 seconds creating final representation with 179 relevant facts, 11 relevant fluents
               0.00 seconds computing LNF
               0.00 seconds building connectivity graph
            2467.12 seconds searching, evaluating 5517244 states, to a max depth of 27
            2467.12 seconds total time
\end{lstlisting}
\subsection*{7.1}
\begin{lstlisting}
ff: parsing domain file
domain 'BELKAN' defined
 ... done.
ff: parsing problem file
problem 'EJ1' defined
 ... done.


no metric specified. plan length assumed.

checking for cyclic := effects --- OK.

ff: search configuration is EHC, if that fails then  best-first on 1*g(s) + 5*h(s) where
    metric is  plan length

Cueing down from goal distance:   21 into depth [1]
                                  20            [1]
                                  19            [1]
                                  18            [1]
                                  17            [1][2][3]
                                  16            [1][2][3][4][5]
                                  15            [1][2]
                                  13            [1]
                                  12            [1][2][3]
                                  11            [1][2][3]
                                  10            [1]
                                   9            [1]
                                   8            [1][2]
                                   7            [1]
                                   6            [1]
                                   5            [1][2][3]
                                   4            [1]
                                   3            [1]
                                   2            [1]
                                   1            [1]
                                   0

ff: found legal plan as follows

step    0: TURN-LEFT ROBOT1 NORTE OESTE
        1: TURN-LEFT ROBOT2 NORTE OESTE
        2: GOTO ROBOT2 OESTE LOC25 LOC24
        3: GOTO ROBOT2 OESTE LOC24 LOC23
        4: TURN-LEFT ROBOT1 OESTE SUR
        5: GOTO ROBOT1 SUR LOC1 LOC6
        6: TURN-LEFT ROBOT1 SUR ESTE
        7: GOTO ROBOT1 ESTE LOC6 LOC7
        8: TURN-RIGHT ROBOT1 ESTE SUR
        9: GOTO ROBOT1 SUR LOC7 LOC12
       10: GOTO ROBOT1 SUR LOC12 LOC17
       11: TURN-RIGHT ROBOT1 SUR OESTE
       12: GOTO ROBOT1 OESTE LOC17 LOC16
       13: TURN-LEFT ROBOT1 OESTE SUR
       14: TURN-LEFT ROBOT1 SUR ESTE
       15: PICK ROBOT1 LOC16 BIKINI
       16: GOTO ROBOT1 ESTE LOC16 LOC17
       17: PUSH-BAG ROBOT1 BIKINI
       18: PICK ROBOT1 LOC17 MANZANA
       19: TURN-RIGHT ROBOT1 ESTE SUR
       20: GOTO-SPECIAL ROBOT1 SUR LOC17 LOC22 AGUA BIKINI
       21: TURN-LEFT ROBOT1 SUR ESTE
       22: GOTO ROBOT1 ESTE LOC22 LOC23
       23: GIVE-COOP ROBOT1 ROBOT2 LOC23 BIKINI
       24: PULL-BAG ROBOT1 BIKINI
       25: PUSH-BAG ROBOT2 BIKINI
       26: GIVE-COOP ROBOT1 ROBOT2 LOC23 MANZANA
       27: GOTO-SPECIAL ROBOT2 OESTE LOC23 LOC22 AGUA BIKINI
       28: GOTO ROBOT2 OESTE LOC22 LOC21
       29: GIVE ROBOT2 LOC21 MANZANA BRUJA


time spent:    0.02 seconds instantiating 1553 easy, 28 hard action templates
               0.00 seconds reachability analysis, yielding 181 facts and 568 actions
               0.00 seconds creating final representation with 176 relevant facts, 11 relevant fluents
               0.00 seconds computing LNF
               0.00 seconds building connectivity graph
               0.00 seconds searching, evaluating 70 states, to a max depth of 5
               0.02 seconds total time
\end{lstlisting}
\subsection*{7.2}
\begin{lstlisting}
ff: parsing domain file
domain 'BELKAN' defined
 ... done.
ff: parsing problem file
problem 'EJ1' defined
 ... done.


no metric specified. plan length assumed.

checking for cyclic := effects --- OK.

ff: search configuration is EHC, if that fails then  best-first on 1*g(s) + 5*h(s) where
    metric is  plan length

Cueing down from goal distance:   37 into depth [1]
                                  36            [1][2][3]
                                  35            [1][2][3][4][5]
                                  34            [1][2]
                                  32            [1]
                                  31            [1][2][3]
                                  30            [1][2][3]
                                  29            [1]
                                  28            [1]
                                  27            [1][2][3]
                                  26            [1][2]
                                  25            [1]
                                  24            [1][2]
                                  23            [1][2]
                                  22            [1][2]
                                  21            [1][2][3]
                                  20            [1]
                                  19            [1]
                                  18            [1][2][3][4][5][6][7][8][9]
                                  17            [1][2][3][4][5][6][7][8][9][10][11][12][13][14][15][16]
                                  16            [1]
                                  15            [1][2][3][4][5][6]
                                  14            [1][2][3][4][5][6][7][8][9][10][11]
                                  13            [1][2][3][4][5][6][7]
                                  12            [1][2][3][4][5][6][7][8][9][10][11][12][13][14][15][16][17][18][19][20][21]
                                  11            [1]
                                  10            [1]
                                   9            [1][2]
                                   8            [1][2][3]
                                   7            [1][2][3][4][5][6][7][8]
                                   6            [1]
                                   5            [1][2]
                                   4            [1][2]
                                   3            [1]
                                   2            [1]
                                   1            [1]
                                   0

ff: found legal plan as follows

step    0: TURN-LEFT ROBOT1 NORTE OESTE
        1: TURN-LEFT ROBOT1 OESTE SUR
        2: GOTO ROBOT1 SUR LOC1 LOC6
        3: TURN-LEFT ROBOT1 SUR ESTE
        4: GOTO ROBOT1 ESTE LOC6 LOC7
        5: TURN-RIGHT ROBOT1 ESTE SUR
        6: GOTO ROBOT1 SUR LOC7 LOC12
        7: GOTO ROBOT1 SUR LOC12 LOC17
        8: TURN-RIGHT ROBOT1 SUR OESTE
        9: GOTO ROBOT1 OESTE LOC17 LOC16
       10: TURN-LEFT ROBOT1 OESTE SUR
       11: TURN-LEFT ROBOT1 SUR ESTE
       12: PICK ROBOT1 LOC16 BIKINI
       13: GOTO ROBOT1 ESTE LOC16 LOC17
       14: PUSH-BAG ROBOT1 BIKINI
       15: PICK ROBOT1 LOC17 MANZANA
       16: TURN-RIGHT ROBOT1 ESTE SUR
       17: GOTO-SPECIAL ROBOT1 SUR LOC17 LOC22 AGUA BIKINI
       18: TURN-LEFT ROBOT1 SUR ESTE
       19: GOTO ROBOT1 ESTE LOC22 LOC23
       20: GOTO ROBOT1 ESTE LOC23 LOC24
       21: GOTO ROBOT1 ESTE LOC24 LOC25
       22: GIVE-COOP ROBOT1 ROBOT2 LOC25 MANZANA
       23: PULL-BAG ROBOT1 BIKINI
       24: PUSH-BAG ROBOT2 MANZANA
       25: GIVE-COOP ROBOT1 ROBOT2 LOC25 BIKINI
       26: GIVE ROBOT2 LOC25 MANZANA PROFESOR
       27: PULL-BAG ROBOT2 MANZANA
       28: TURN-LEFT ROBOT2 NORTE OESTE
       29: GOTO ROBOT2 OESTE LOC25 LOC24
       30: GOTO ROBOT2 OESTE LOC24 LOC23
       31: GOTO-SPECIAL ROBOT2 OESTE LOC23 LOC22 AGUA BIKINI
       32: TURN-RIGHT ROBOT2 OESTE NORTE
       33: GOTO ROBOT2 NORTE LOC22 LOC17
       34: GOTO ROBOT2 NORTE LOC17 LOC12
       35: TURN-RIGHT ROBOT2 NORTE ESTE
       36: GOTO ROBOT2 ESTE LOC12 LOC13
       37: GOTO ROBOT2 ESTE LOC13 LOC14
       38: GOTO ROBOT2 ESTE LOC14 LOC15
       39: TURN-RIGHT ROBOT2 ESTE SUR
       40: TURN-RIGHT ROBOT2 SUR OESTE
       41: TURN-RIGHT ROBOT2 OESTE NORTE
       42: GOTO ROBOT2 NORTE LOC15 LOC10
       43: GOTO ROBOT2 NORTE LOC10 LOC5
       44: TURN-LEFT ROBOT2 NORTE OESTE
       45: TURN-LEFT ROBOT2 OESTE SUR
       46: GIVE ROBOT2 LOC5 MANZANA PRINCIPE
       47: GOTO ROBOT2 SUR LOC5 LOC10
       48: GOTO ROBOT2 SUR LOC10 LOC15
       49: TURN-RIGHT ROBOT1 ESTE SUR
       50: TURN-RIGHT ROBOT1 SUR OESTE
       51: GOTO ROBOT1 OESTE LOC25 LOC24
       52: GOTO ROBOT1 OESTE LOC24 LOC23
       53: TURN-RIGHT ROBOT2 SUR OESTE
       54: GOTO ROBOT2 OESTE LOC15 LOC14
       55: GOTO ROBOT2 OESTE LOC14 LOC13
       56: PICK ROBOT1 LOC23 MANZANA
       57: GOTO ROBOT2 OESTE LOC13 LOC12
       58: TURN-RIGHT ROBOT2 OESTE NORTE
       59: GOTO ROBOT2 NORTE LOC12 LOC7
       60: GOTO ROBOT2 NORTE LOC7 LOC2
       61: TURN-RIGHT ROBOT2 NORTE ESTE
       62: TURN-RIGHT ROBOT2 ESTE SUR
       63: GOTO ROBOT2 SUR LOC2 LOC7
       64: GOTO ROBOT2 SUR LOC7 LOC12
       65: GOTO ROBOT2 SUR LOC12 LOC17
       66: GOTO-SPECIAL ROBOT2 SUR LOC17 LOC22 AGUA BIKINI
       67: TURN-LEFT ROBOT2 SUR ESTE
       68: GOTO ROBOT2 ESTE LOC22 LOC23
       69: TURN-RIGHT ROBOT2 ESTE SUR
       70: TURN-RIGHT ROBOT2 SUR OESTE
       71: GIVE-COOP ROBOT1 ROBOT2 LOC23 MANZANA
       72: GOTO-SPECIAL ROBOT2 OESTE LOC23 LOC22 AGUA BIKINI
       73: TURN-LEFT ROBOT2 OESTE SUR
       74: TURN-LEFT ROBOT2 SUR ESTE
       75: TURN-LEFT ROBOT2 ESTE NORTE
       76: GOTO ROBOT2 NORTE LOC22 LOC17
       77: GOTO ROBOT2 NORTE LOC17 LOC12
       78: GOTO ROBOT2 NORTE LOC12 LOC7
       79: GOTO ROBOT2 NORTE LOC7 LOC2
       80: TURN-RIGHT ROBOT2 NORTE ESTE
       81: TURN-RIGHT ROBOT2 ESTE SUR
       82: TURN-RIGHT ROBOT2 SUR OESTE
       83: GOTO ROBOT2 OESTE LOC2 LOC1
       84: TURN-LEFT ROBOT2 OESTE SUR
       85: GIVE ROBOT2 LOC1 MANZANA PRINCESA
       86: GOTO ROBOT2 SUR LOC1 LOC6
       87: TURN-LEFT ROBOT1 OESTE SUR
       88: TURN-LEFT ROBOT2 SUR ESTE
       89: GOTO ROBOT2 ESTE LOC6 LOC7
       90: TURN-LEFT ROBOT1 SUR ESTE
       91: GOTO ROBOT1 ESTE LOC23 LOC24
       92: TURN-RIGHT ROBOT1 ESTE SUR
       93: TURN-RIGHT ROBOT1 SUR OESTE
       94: PICK ROBOT1 LOC24 BIKINI
       95: GOTO ROBOT1 OESTE LOC24 LOC23
       96: GOTO-SPECIAL ROBOT1 OESTE LOC23 LOC22 AGUA BIKINI
       97: TURN-RIGHT ROBOT1 OESTE NORTE
       98: GOTO ROBOT1 NORTE LOC22 LOC17
       99: GOTO ROBOT1 NORTE LOC17 LOC12
      100: PUSH-BAG ROBOT1 BIKINI
      101: GOTO ROBOT1 NORTE LOC12 LOC7
      102: PICK ROBOT1 LOC7 OSCAR
      103: GIVE-COOP ROBOT1 ROBOT2 LOC7 OSCAR
      104: TURN-RIGHT ROBOT2 ESTE SUR
      105: GOTO ROBOT2 SUR LOC7 LOC12
      106: TURN-LEFT ROBOT2 SUR ESTE
      107: GOTO ROBOT2 ESTE LOC12 LOC13
      108: TURN-RIGHT ROBOT2 ESTE SUR
      109: GIVE ROBOT2 LOC13 OSCAR LEONARDO
      110: TURN-RIGHT ROBOT1 NORTE ESTE
      111: TURN-RIGHT ROBOT1 ESTE SUR
      112: GOTO ROBOT1 SUR LOC7 LOC12
      113: TURN-LEFT ROBOT1 SUR ESTE
      114: GOTO ROBOT1 ESTE LOC12 LOC13
      115: PICK ROBOT1 LOC13 MANZANA
      116: GIVE-COOP ROBOT1 ROBOT2 LOC13 MANZANA
      117: GOTO ROBOT2 SUR LOC13 LOC18
      118: TURN-RIGHT ROBOT2 SUR OESTE
      119: GOTO ROBOT2 OESTE LOC18 LOC17
      120: TURN-LEFT ROBOT2 OESTE SUR
      121: GOTO-SPECIAL ROBOT2 SUR LOC17 LOC22 AGUA BIKINI
      122: TURN-RIGHT ROBOT2 SUR OESTE
      123: GOTO ROBOT2 OESTE LOC22 LOC21
      124: GIVE ROBOT2 LOC21 MANZANA BRUJA


time spent:    0.02 seconds instantiating 1553 easy, 28 hard action templates
               0.00 seconds reachability analysis, yielding 181 facts and 568 actions
               0.00 seconds creating final representation with 176 relevant facts, 15 relevant fluents
               0.01 seconds computing LNF
               0.00 seconds building connectivity graph
               0.46 seconds searching, evaluating 10897 states, to a max depth of 21
               0.49 seconds total time
\end{lstlisting}
\subsection*{7.3}
\begin{lstlisting}
ff: parsing domain file
domain 'BELKAN' defined
 ... done.
ff: parsing problem file
problem 'EJ1' defined
 ... done.


no metric specified. plan length assumed.

checking for cyclic := effects --- OK.

ff: search configuration is EHC, if that fails then  best-first on 1*g(s) + 5*h(s) where
    metric is  plan length

Cueing down from goal distance:   10 into depth [1]
                                   9            [1]
                                   8            [1][2][3][4]
                                   6            [1]
                                   5            [1]
                                   4            [1]
                                   3            [1]
                                   2            [1][2][3][4][5][6]
                                   1            [1] --- pruning stopped --- [1][2][3][4][5][6][7][8][9][10][11][12][13][14][15][16][17][18][19][20][21][22][23][24][25][26][27]
                                   0            

ff: found legal plan as follows

step    0: TURN-LEFT ROBOT1 NORTE OESTE
        1: TURN-LEFT ROBOT1 OESTE SUR
        2: GOTO ROBOT1 SUR LOC1 LOC6
        3: TURN-LEFT ROBOT1 SUR ESTE
        4: GOTO ROBOT1 ESTE LOC6 LOC7
        5: TURN-RIGHT ROBOT1 ESTE SUR
        6: PICK ROBOT1 LOC7 OSCAR
        7: GOTO ROBOT1 SUR LOC7 LOC12
        8: TURN-LEFT ROBOT1 SUR ESTE
        9: GOTO ROBOT1 ESTE LOC12 LOC13
       10: GIVE ROBOT1 LOC13 OSCAR LEONARDO
       11: GOTO ROBOT1 ESTE LOC13 LOC14
       12: TURN-RIGHT ROBOT1 ESTE SUR
       13: TURN-RIGHT ROBOT1 SUR OESTE
       14: PICK ROBOT1 LOC14 OSCAR
       15: GOTO ROBOT1 OESTE LOC14 LOC13
       16: TURN-LEFT ROBOT2 NORTE OESTE
       17: GOTO ROBOT1 OESTE LOC13 LOC12
       18: GOTO ROBOT2 OESTE LOC25 LOC24
       19: GOTO ROBOT1 OESTE LOC12 LOC11
       20: TURN-LEFT ROBOT1 OESTE SUR
       21: DROP ROBOT1 LOC11 OSCAR
       22: GOTO ROBOT1 SUR LOC11 LOC16
       23: TURN-LEFT ROBOT1 SUR ESTE
       24: PICK ROBOT1 LOC16 BIKINI
       25: GOTO ROBOT1 ESTE LOC16 LOC17
       26: TURN-RIGHT ROBOT1 ESTE SUR
       27: PUSH-BAG ROBOT1 BIKINI
       28: GOTO-SPECIAL ROBOT1 SUR LOC17 LOC22 AGUA BIKINI
       29: TURN-LEFT ROBOT1 SUR ESTE
       30: PICK ROBOT1 LOC22 ZAPATILLA
       31: GOTO-SPECIAL ROBOT1 ESTE LOC22 LOC23 BOSQUE ZAPATILLA
       32: GOTO ROBOT1 ESTE LOC23 LOC24
       33: DROP ROBOT1 LOC24 ZAPATILLA
       34: PICK ROBOT2 LOC24 ZAPATILLA
       35: GOTO-SPECIAL ROBOT2 OESTE LOC24 LOC23 BOSQUE ZAPATILLA
       36: TURN-LEFT ROBOT2 OESTE SUR
       37: PUSH-BAG ROBOT2 ZAPATILLA
       38: TURN-LEFT ROBOT2 SUR ESTE
       39: PICK ROBOT2 LOC23 MANZANA
       40: GOTO ROBOT2 ESTE LOC23 LOC24
       41: GOTO ROBOT2 ESTE LOC24 LOC25
       42: GIVE ROBOT2 LOC25 MANZANA PROFESOR
     

time spent:    0.00 seconds instantiating 2086 easy, 0 hard action templates
               0.00 seconds reachability analysis, yielding 190 facts and 534 actions
               0.00 seconds creating final representation with 179 relevant facts, 11 relevant fluents
               0.00 seconds computing LNF
               0.00 seconds building connectivity graph
            2467.12 seconds searching, evaluating 5517244 states, to a max depth of 27
            2467.12 seconds total time
\end{lstlisting}
	
\end{document}